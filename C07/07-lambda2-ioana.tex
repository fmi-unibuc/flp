\documentclass[xcolor=pdftex,romanian,colorlinks]{beamer}

 \usetheme{Median}
%% General document %%%%%%%%%%%%%%%%%%%%%%%%%%%%%%%%%%

\usepackage[romanian]{babel}

\setbeamertemplate{footline}[frame number]
%%
%% `BeamerColor.sty',
%% 
%%   Dieser Text ist urheberrechtlich gesch�tzt
%%   Er stellt einen Auszug eines von mir erstellten Referates dar
%%   und darf nicht gewerblich genutzt werden
%%   die private bzw. Studiums bezogen Nutzung ist frei
%% 
%% Autor: Sascha Frank 
%% 
%% www.informatik.uni-freiburg.de/~frank/
%% 
%% \usetheme{Was_auch_immer}
%% \usecolortheme[named=Farbe]{structure}
%%
%% Beispielsweise das Usetheme Berkeley in rot anstatt dem �blichen blau:
%%
%% \usetheme{Berkeley}
%% \usecolortheme[named=red]{structure}
%%  
%%   
%%   
%%        
%% 
%% 
%%
%% 
\NeedsTeXFormat{LaTeX2e}
\ProvidesPackage{BeamerColor}[08/01/2008]
\RequirePackage{xcolor}

\definecolor{AliceBlue}{rgb}{0.94,0.97,1}
\definecolor{BlueViolet}{rgb}{0.54,0.17,0.88}
\definecolor{CadetBlue}{rgb}{0.37,0.62,0.63}
\definecolor{CadetBlue1}{rgb}{0.59,0.96,1}
\definecolor{CadetBlue2}{rgb}{0.55,0.89,0.93}
\definecolor{CadetBlue3}{rgb}{0.48,0.77,0.8}
\definecolor{CadetBlue4}{rgb}{0.32,0.52,0.54}
\definecolor{CornflowerBlue}{rgb}{0.39,0.58,0.93}
\definecolor{DarkSlateBlue}{rgb}{0.28,0.24,0.54}
\definecolor{DarkTurquoise}{rgb}{0,0.8,0.82}
\definecolor{DeepSkyBlue}{rgb}{0,0.75,1}
\definecolor{DeepSkyBlue1}{rgb}{0,0.75,1}
\definecolor{DeepSkyBlue2}{rgb}{0,0.7,0.93}
\definecolor{DeepSkyBlue3}{rgb}{0,0.6,0.8}
\definecolor{DeepSkyBlue4}{rgb}{0,0.41,0.54}
\definecolor{DodgerBlue}{rgb}{0.12,0.56,1}
\definecolor{DodgerBlue1}{rgb}{0.12,0.56,1}
\definecolor{DodgerBlue2}{rgb}{0.11,0.52,0.93}
\definecolor{DodgerBlue3}{rgb}{0.09,0.45,0.8}
\definecolor{DodgerBlue4}{rgb}{0.06,0.3,0.54}
\definecolor{LightBlue}{rgb}{0.68,0.84,0.9}
\definecolor{LightBlue1}{rgb}{0.75,0.93,1}
\definecolor{LightBlue2}{rgb}{0.7,0.87,0.93}
\definecolor{LightBlue3}{rgb}{0.6,0.75,0.8}
\definecolor{LightBlue4}{rgb}{0.41,0.51,0.54}
\definecolor{LightCyan}{rgb}{0.88,1,1}
\definecolor{LightCyan1}{rgb}{0.88,1,1}
\definecolor{LightCyan2}{rgb}{0.82,0.93,0.93}
\definecolor{LightCyan3}{rgb}{0.7,0.8,0.8}
\definecolor{LightCyan4}{rgb}{0.48,0.54,0.54}
\definecolor{LightSkyBlue}{rgb}{0.53,0.8,0.98}
\definecolor{LightSkyBlue1}{rgb}{0.69,0.88,1}
\definecolor{LightSkyBlue2}{rgb}{0.64,0.82,0.93}
\definecolor{LightSkyBlue3}{rgb}{0.55,0.71,0.8}
\definecolor{LightSkyBlue4}{rgb}{0.38,0.48,0.54}
\definecolor{LightSlateBlue}{rgb}{0.52,0.44,1}
\definecolor{LightSteelBlue}{rgb}{0.69,0.77,0.87}
\definecolor{LightSteelBlue1}{rgb}{0.79,0.88,1}
\definecolor{LightSteelBlue2}{rgb}{0.73,0.82,0.93}
\definecolor{LightSteelBlue3}{rgb}{0.63,0.71,0.8}
\definecolor{LightSteelBlue4}{rgb}{0.43,0.48,0.54}
\definecolor{MediumAquamarine}{rgb}{0.4,0.8,0.66}
\definecolor{MediumBlue}{rgb}{0,0,0.8}
\definecolor{MediumSlateBlue}{rgb}{0.48,0.41,0.93}
\definecolor{MediumTurquoise}{rgb}{0.28,0.82,0.8}
\definecolor{MidnightBlue}{rgb}{0.1,0.1,0.44}
\definecolor{NavyBlue}{rgb}{0,0,0.5}
\definecolor{PaleTurquoise}{rgb}{0.68,0.93,0.93}
\definecolor{PaleTurquoise1}{rgb}{0.73,1,1}
\definecolor{PaleTurquoise2}{rgb}{0.68,0.93,0.93}
\definecolor{PaleTurquoise3}{rgb}{0.59,0.8,0.8}
\definecolor{PaleTurquoise4}{rgb}{0.4,0.54,0.54}
\definecolor{PowderBlue}{rgb}{0.69,0.88,0.9}
\definecolor{RoyalBlue}{rgb}{0.25,0.41,0.88}
\definecolor{RoyalBlue1}{rgb}{0.28,0.46,1}
\definecolor{RoyalBlue2}{rgb}{0.26,0.43,0.93}
\definecolor{RoyalBlue3}{rgb}{0.23,0.37,0.8}
\definecolor{RoyalBlue4}{rgb}{0.15,0.25,0.54}
\definecolor{SkyBlue}{rgb}{0.53,0.8,0.92}
\definecolor{SkyBlue1}{rgb}{0.53,0.8,1}
\definecolor{SkyBlue2}{rgb}{0.49,0.75,0.93}
\definecolor{SkyBlue3}{rgb}{0.42,0.65,0.8}
\definecolor{SkyBlue4}{rgb}{0.29,0.44,0.54}
\definecolor{SlateBlue}{rgb}{0.41,0.35,0.8}
\definecolor{SlateBlue1}{rgb}{0.51,0.43,1}
\definecolor{SlateBlue2}{rgb}{0.48,0.4,0.93}
\definecolor{SlateBlue3}{rgb}{0.41,0.35,0.8}
\definecolor{SlateBlue4}{rgb}{0.28,0.23,0.54}
\definecolor{SteelBlue}{rgb}{0.27,0.51,0.7}
\definecolor{SteelBlue1}{rgb}{0.39,0.72,1}
\definecolor{SteelBlue2}{rgb}{0.36,0.67,0.93}
\definecolor{SteelBlue3}{rgb}{0.31,0.58,0.8}
\definecolor{SteelBlue4}{rgb}{0.21,0.39,0.54}
\definecolor{aquamarine}{rgb}{0.5,1,0.83}
\definecolor{aquamarine1}{rgb}{0.5,1,0.83}
\definecolor{aquamarine2}{rgb}{0.46,0.93,0.77}
\definecolor{aquamarine3}{rgb}{0.4,0.8,0.66}
\definecolor{aquamarine4}{rgb}{0.27,0.54,0.45}
\definecolor{azure}{rgb}{0.94,1,1}
\definecolor{azure1}{rgb}{0.94,1,1}
\definecolor{azure2}{rgb}{0.88,0.93,0.93}
\definecolor{azure3}{rgb}{0.75,0.8,0.8}
\definecolor{azure4}{rgb}{0.51,0.54,0.54}
\definecolor{blue}{rgb}{0,0,1}
\definecolor{blue1}{rgb}{0,0,1}
\definecolor{blue2}{rgb}{0,0,0.93}
\definecolor{blue3}{rgb}{0,0,0.8}
\definecolor{blue4}{rgb}{0,0,0.54}
\definecolor{cyan}{rgb}{0,1,1}
\definecolor{cyan1}{rgb}{0,1,1}
\definecolor{cyan2}{rgb}{0,0.93,0.93}
\definecolor{cyan3}{rgb}{0,0.8,0.8}
\definecolor{cyan4}{rgb}{0,0.54,0.54}
\definecolor{navy}{rgb}{0,0,0.5}
\definecolor{turquoise}{rgb}{0.25,0.88,0.81}
\definecolor{turquoise1}{rgb}{0,0.96,1}
\definecolor{turquoise2}{rgb}{0,0.89,0.93}
\definecolor{turquoise3}{rgb}{0,0.77,0.8}
\definecolor{turquoise4}{rgb}{0,0.52,0.54}
\definecolor{RosyBrown}{rgb}{0.73,0.56,0.56}
\definecolor{RosyBrown1}{rgb}{1,0.75,0.75}
\definecolor{RosyBrown2}{rgb}{0.93,0.7,0.7}
\definecolor{RosyBrown3}{rgb}{0.8,0.61,0.61}
\definecolor{RosyBrown4}{rgb}{0.54,0.41,0.41}
\definecolor{SaddleBrown}{rgb}{0.54,0.27,0.07}
\definecolor{SandyBrown}{rgb}{0.95,0.64,0.38}
\definecolor{beige}{rgb}{0.96,0.96,0.86}
\definecolor{brown}{rgb}{0.64,0.16,0.16}
\definecolor{brown1}{rgb}{1,0.25,0.25}
\definecolor{brown2}{rgb}{0.93,0.23,0.23}
\definecolor{brown3}{rgb}{0.8,0.2,0.2}
\definecolor{brown4}{rgb}{0.54,0.14,0.14}
\definecolor{burlywood}{rgb}{0.87,0.72,0.53}
\definecolor{burlywood1}{rgb}{1,0.82,0.61}
\definecolor{burlywood2}{rgb}{0.93,0.77,0.57}
\definecolor{burlywood3}{rgb}{0.8,0.66,0.49}
\definecolor{burlywood4}{rgb}{0.54,0.45,0.33}
\definecolor{chocolate}{rgb}{0.82,0.41,0.12}
\definecolor{chocolate1}{rgb}{1,0.5,0.14}
\definecolor{chocolate2}{rgb}{0.93,0.46,0.13}
\definecolor{chocolate3}{rgb}{0.8,0.4,0.11}
\definecolor{chocolate4}{rgb}{0.54,0.27,0.07}
\definecolor{peru}{rgb}{0.8,0.52,0.25}
\definecolor{tan}{rgb}{0.82,0.7,0.55}
\definecolor{tan1}{rgb}{1,0.64,0.31}
\definecolor{tan2}{rgb}{0.93,0.6,0.29}
\definecolor{tan3}{rgb}{0.8,0.52,0.25}
\definecolor{tan4}{rgb}{0.54,0.35,0.17}
\definecolor{DarkSlateGray}{rgb}{0.18,0.31,0.31}
\definecolor{DarkSlateGray1}{rgb}{0.59,1,1}
\definecolor{DarkSlateGray2}{rgb}{0.55,0.93,0.93}
\definecolor{DarkSlateGray3}{rgb}{0.47,0.8,0.8}
\definecolor{DarkSlateGray4}{rgb}{0.32,0.54,0.54}
\definecolor{DarkSlateGrey}{rgb}{0.18,0.31,0.31}
\definecolor{DimGray}{rgb}{0.41,0.41,0.41}
\definecolor{DimGrey}{rgb}{0.41,0.41,0.41}
\definecolor{LightGray}{rgb}{0.82,0.82,0.82}
\definecolor{LightGrey}{rgb}{0.82,0.82,0.82}
\definecolor{LightSlateGray}{rgb}{0.46,0.53,0.6}
\definecolor{LightSlateGrey}{rgb}{0.46,0.53,0.6}
\definecolor{SlateGray}{rgb}{0.44,0.5,0.56}
\definecolor{SlateGray1}{rgb}{0.77,0.88,1}
\definecolor{SlateGray2}{rgb}{0.72,0.82,0.93}
\definecolor{SlateGray3}{rgb}{0.62,0.71,0.8}
\definecolor{SlateGray4}{rgb}{0.42,0.48,0.54}
\definecolor{SlateGrey}{rgb}{0.44,0.5,0.56}
\definecolor{gray}{rgb}{0.74,0.74,0.74}
\definecolor{gray0}{rgb}{0,0,0}
\definecolor{gray1}{rgb}{0.01,0.01,0.01}
\definecolor{gray10}{rgb}{0.1,0.1,0.1}
\definecolor{DarkGreen}{rgb}{0,0.39,0}
\definecolor{DarkKhaki}{rgb}{0.74,0.71,0.42}
\definecolor{DarkOliveGreen}{rgb}{0.33,0.42,0.18}
\definecolor{DarkOliveGreen1}{rgb}{0.79,1,0.44}
\definecolor{DarkOliveGreen2}{rgb}{0.73,0.93,0.41}
\definecolor{DarkOliveGreen3}{rgb}{0.63,0.8,0.35}
\definecolor{DarkOliveGreen4}{rgb}{0.43,0.54,0.24}
\definecolor{DarkSeaGreen}{rgb}{0.56,0.73,0.56}
\definecolor{DarkSeaGreen1}{rgb}{0.75,1,0.75}
\definecolor{DarkSeaGreen2}{rgb}{0.7,0.93,0.7}
\definecolor{DarkSeaGreen3}{rgb}{0.61,0.8,0.61}
\definecolor{DarkSeaGreen4}{rgb}{0.41,0.54,0.41}
\definecolor{ForestGreen}{rgb}{0.13,0.54,0.13}
\definecolor{GreenYellow}{rgb}{0.68,1,0.18}
\definecolor{LawnGreen}{rgb}{0.48,0.98,0}
\definecolor{LightSeaGreen}{rgb}{0.13,0.7,0.66}
\definecolor{LimeGreen}{rgb}{0.2,0.8,0.2}
\definecolor{MediumSeaGreen}{rgb}{0.23,0.7,0.44}
\definecolor{MediumSpringGreen}{rgb}{0,0.98,0.6}
\definecolor{MintCream}{rgb}{0.96,1,0.98}
\definecolor{OliveDrab}{rgb}{0.42,0.55,0.14}
\definecolor{OliveDrab1}{rgb}{0.75,1,0.24}
\definecolor{OliveDrab2}{rgb}{0.7,0.93,0.23}
\definecolor{OliveDrab3}{rgb}{0.6,0.8,0.2}
\definecolor{OliveDrab4}{rgb}{0.41,0.54,0.13}
\definecolor{PaleGreen}{rgb}{0.59,0.98,0.59}
\definecolor{PaleGreen1}{rgb}{0.6,1,0.6}
\definecolor{PaleGreen2}{rgb}{0.56,0.93,0.56}
\definecolor{PaleGreen3}{rgb}{0.48,0.8,0.48}
\definecolor{PaleGreen4}{rgb}{0.33,0.54,0.33}
\definecolor{SeaGreen}{rgb}{0.18,0.54,0.34}
\definecolor{SeaGreen1}{rgb}{0.33,1,0.62}
\definecolor{SeaGreen2}{rgb}{0.3,0.93,0.58}
\definecolor{SeaGreen3}{rgb}{0.26,0.8,0.5}
\definecolor{SeaGreen4}{rgb}{0.18,0.54,0.34}
\definecolor{SpringGreen}{rgb}{0,1,0.5}
\definecolor{SpringGreen1}{rgb}{0,1,0.5}
\definecolor{SpringGreen2}{rgb}{0,0.93,0.46}
\definecolor{SpringGreen3}{rgb}{0,0.8,0.4}
\definecolor{SpringGreen4}{rgb}{0,0.54,0.27}
\definecolor{YellowGreen}{rgb}{0.6,0.8,0.2}
\definecolor{chartreuse}{rgb}{0.5,1,0}
\definecolor{chartreuse1}{rgb}{0.5,1,0}
\definecolor{chartreuse2}{rgb}{0.46,0.93,0}
\definecolor{chartreuse3}{rgb}{0.4,0.8,0}
\definecolor{chartreuse4}{rgb}{0.27,0.54,0}
\definecolor{green}{rgb}{0,1,0}
\definecolor{green1}{rgb}{0,1,0}
\definecolor{green2}{rgb}{0,0.93,0}
\definecolor{green3}{rgb}{0,0.8,0}
\definecolor{green4}{rgb}{0,0.54,0}
\definecolor{khaki}{rgb}{0.94,0.9,0.55}
\definecolor{khaki1}{rgb}{1,0.96,0.56}
\definecolor{khaki2}{rgb}{0.93,0.9,0.52}
\definecolor{khaki3}{rgb}{0.8,0.77,0.45}
\definecolor{khaki4}{rgb}{0.54,0.52,0.3}
\definecolor{DarkOrange}{rgb}{1,0.55,0}
\definecolor{DarkOrange1}{rgb}{1,0.5,0}
\definecolor{DarkOrange2}{rgb}{0.93,0.46,0}
\definecolor{DarkOrange3}{rgb}{0.8,0.4,0}
\definecolor{DarkOrange4}{rgb}{0.54,0.27,0}
\definecolor{DarkSalmon}{rgb}{0.91,0.59,0.48}
\definecolor{LightCoral}{rgb}{0.94,0.5,0.5}
\definecolor{LightSalmon}{rgb}{1,0.63,0.48}
\definecolor{LightSalmon1}{rgb}{1,0.63,0.48}
\definecolor{LightSalmon2}{rgb}{0.93,0.58,0.45}
\definecolor{LightSalmon3}{rgb}{0.8,0.5,0.38}
\definecolor{LightSalmon4}{rgb}{0.54,0.34,0.26}
\definecolor{PeachPuff}{rgb}{1,0.85,0.72}
\definecolor{PeachPuff1}{rgb}{1,0.85,0.72}
\definecolor{PeachPuff2}{rgb}{0.93,0.79,0.68}
\definecolor{PeachPuff3}{rgb}{0.8,0.68,0.58}
\definecolor{PeachPuff4}{rgb}{0.54,0.46,0.39}
\definecolor{bisque}{rgb}{1,0.89,0.77}
\definecolor{bisque1}{rgb}{1,0.89,0.77}
\definecolor{bisque2}{rgb}{0.93,0.83,0.71}
\definecolor{bisque3}{rgb}{0.8,0.71,0.62}
\definecolor{bisque4}{rgb}{0.54,0.49,0.42}
\definecolor{coral}{rgb}{1,0.5,0.31}
\definecolor{coral1}{rgb}{1,0.45,0.34}
\definecolor{coral2}{rgb}{0.93,0.41,0.31}
\definecolor{coral3}{rgb}{0.8,0.36,0.27}
\definecolor{coral4}{rgb}{0.54,0.24,0.18}
\definecolor{honeydew}{rgb}{0.94,1,0.94}
\definecolor{honeydew1}{rgb}{0.94,1,0.94}
\definecolor{honeydew2}{rgb}{0.88,0.93,0.88}
\definecolor{honeydew3}{rgb}{0.75,0.8,0.75}
\definecolor{honeydew4}{rgb}{0.51,0.54,0.51}
\definecolor{orange}{rgb}{1,0.64,0}
\definecolor{orange1}{rgb}{1,0.64,0}
\definecolor{orange2}{rgb}{0.93,0.6,0}
\definecolor{orange3}{rgb}{0.8,0.52,0}
\definecolor{orange4}{rgb}{0.54,0.35,0}
\definecolor{salmon}{rgb}{0.98,0.5,0.45}
\definecolor{salmon1}{rgb}{1,0.55,0.41}
\definecolor{salmon2}{rgb}{0.93,0.51,0.38}
\definecolor{salmon3}{rgb}{0.8,0.44,0.33}
\definecolor{salmon4}{rgb}{0.54,0.3,0.22}
\definecolor{sienna}{rgb}{0.63,0.32,0.18}
\definecolor{sienna1}{rgb}{1,0.51,0.28}
\definecolor{sienna2}{rgb}{0.93,0.47,0.26}
\definecolor{sienna3}{rgb}{0.8,0.41,0.22}
\definecolor{sienna4}{rgb}{0.54,0.28,0.15}
\definecolor{DeepPink}{rgb}{1,0.08,0.57}
\definecolor{DeepPink1}{rgb}{1,0.08,0.57}
\definecolor{DeepPink2}{rgb}{0.93,0.07,0.54}
\definecolor{DeepPink3}{rgb}{0.8,0.06,0.46}
\definecolor{DeepPink4}{rgb}{0.54,0.04,0.31}
\definecolor{HotPink}{rgb}{1,0.41,0.7}
\definecolor{HotPink1}{rgb}{1,0.43,0.7}
\definecolor{HotPink2}{rgb}{0.93,0.41,0.65}
\definecolor{HotPink3}{rgb}{0.8,0.38,0.56}
\definecolor{HotPink4}{rgb}{0.54,0.23,0.38}
\definecolor{IndianRed}{rgb}{0.8,0.36,0.36}
\definecolor{IndianRed1}{rgb}{1,0.41,0.41}
\definecolor{IndianRed2}{rgb}{0.93,0.39,0.39}
\definecolor{IndianRed3}{rgb}{0.8,0.33,0.33}
\definecolor{IndianRed4}{rgb}{0.54,0.23,0.23}
\definecolor{LightPink}{rgb}{1,0.71,0.75}
\definecolor{LightPink1}{rgb}{1,0.68,0.72}
\definecolor{LightPink2}{rgb}{0.93,0.63,0.68}
\definecolor{LightPink3}{rgb}{0.8,0.55,0.58}
\definecolor{LightPink4}{rgb}{0.54,0.37,0.39}
\definecolor{MediumVioletRed}{rgb}{0.78,0.08,0.52}
\definecolor{MistyRose}{rgb}{1,0.89,0.88}
\definecolor{MistyRose1}{rgb}{1,0.89,0.88}
\definecolor{MistyRose2}{rgb}{0.93,0.83,0.82}
\definecolor{MistyRose3}{rgb}{0.8,0.71,0.71}
\definecolor{MistyRose4}{rgb}{0.54,0.49,0.48}
\definecolor{OrangeRed}{rgb}{1,0.27,0}
\definecolor{OrangeRed1}{rgb}{1,0.27,0}
\definecolor{OrangeRed2}{rgb}{0.93,0.25,0}
\definecolor{OrangeRed3}{rgb}{0.8,0.21,0}
\definecolor{OrangeRed4}{rgb}{0.54,0.14,0}
\definecolor{PaleVioletRed}{rgb}{0.86,0.44,0.57}
\definecolor{PaleVioletRed1}{rgb}{1,0.51,0.67}
\definecolor{PaleVioletRed2}{rgb}{0.93,0.47,0.62}
\definecolor{PaleVioletRed3}{rgb}{0.8,0.41,0.54}
\definecolor{PaleVioletRed4}{rgb}{0.54,0.28,0.36}
\definecolor{VioletRed}{rgb}{0.81,0.13,0.56}
\definecolor{VioletRed1}{rgb}{1,0.24,0.59}
\definecolor{VioletRed2}{rgb}{0.93,0.23,0.55}
\definecolor{VioletRed3}{rgb}{0.8,0.2,0.47}
\definecolor{VioletRed4}{rgb}{0.54,0.13,0.32}
\definecolor{firebrick}{rgb}{0.7,0.13,0.13}
\definecolor{firebrick1}{rgb}{1,0.19,0.19}
\definecolor{firebrick2}{rgb}{0.93,0.17,0.17}
\definecolor{firebrick3}{rgb}{0.8,0.15,0.15}
\definecolor{firebrick4}{rgb}{0.54,0.1,0.1}
\definecolor{pink}{rgb}{1,0.75,0.79}
\definecolor{pink1}{rgb}{1,0.71,0.77}
\definecolor{pink2}{rgb}{0.93,0.66,0.72}
\definecolor{pink3}{rgb}{0.8,0.57,0.62}
\definecolor{pink4}{rgb}{0.54,0.39,0.42}
\definecolor{red}{rgb}{1,0,0}
\definecolor{red1}{rgb}{1,0,0}
\definecolor{red2}{rgb}{0.93,0,0}
\definecolor{red3}{rgb}{0.8,0,0}
\definecolor{red4}{rgb}{0.54,0,0}
\definecolor{tomato}{rgb}{1,0.39,0.28}
\definecolor{tomato1}{rgb}{1,0.39,0.28}
\definecolor{tomato2}{rgb}{0.93,0.36,0.26}
\definecolor{tomato3}{rgb}{0.8,0.31,0.22}
\definecolor{tomato4}{rgb}{0.54,0.21,0.15}
\definecolor{DarkOrchid}{rgb}{0.6,0.2,0.8}
\definecolor{DarkOrchid1}{rgb}{0.75,0.24,1}
\definecolor{DarkOrchid2}{rgb}{0.7,0.23,0.93}
\definecolor{DarkOrchid3}{rgb}{0.6,0.2,0.8}
\definecolor{DarkOrchid4}{rgb}{0.41,0.13,0.54}
\definecolor{DarkViolet}{rgb}{0.58,0,0.82}
\definecolor{LavenderBlush}{rgb}{1,0.94,0.96}
\definecolor{LavenderBlush1}{rgb}{1,0.94,0.96}
\definecolor{LavenderBlush2}{rgb}{0.93,0.88,0.89}
\definecolor{LavenderBlush3}{rgb}{0.8,0.75,0.77}
\definecolor{LavenderBlush4}{rgb}{0.54,0.51,0.52}
\definecolor{MediumOrchid}{rgb}{0.73,0.33,0.82}
\definecolor{MediumOrchid1}{rgb}{0.88,0.4,1}
\definecolor{MediumOrchid2}{rgb}{0.82,0.37,0.93}
\definecolor{MediumOrchid3}{rgb}{0.7,0.32,0.8}
\definecolor{MediumOrchid4}{rgb}{0.48,0.21,0.54}
\definecolor{MediumPurple}{rgb}{0.57,0.44,0.86}
\definecolor{MediumPurple1}{rgb}{0.67,0.51,1}
\definecolor{MediumPurple2}{rgb}{0.62,0.47,0.93}
\definecolor{MediumPurple3}{rgb}{0.54,0.41,0.8}
\definecolor{MediumPurple4}{rgb}{0.36,0.28,0.54}
\definecolor{lavender}{rgb}{0.9,0.9,0.98}
\definecolor{magenta}{rgb}{1,0,1}
\definecolor{magenta1}{rgb}{1,0,1}
\definecolor{magenta2}{rgb}{0.93,0,0.93}
\definecolor{magenta3}{rgb}{0.8,0,0.8}
\definecolor{magenta4}{rgb}{0.54,0,0.54}
\definecolor{maroon}{rgb}{0.69,0.19,0.38}
\definecolor{maroon1}{rgb}{1,0.2,0.7}
\definecolor{maroon2}{rgb}{0.93,0.19,0.65}
\definecolor{maroon3}{rgb}{0.8,0.16,0.56}
\definecolor{maroon4}{rgb}{0.54,0.11,0.38}
\definecolor{orchid}{rgb}{0.85,0.44,0.84}
\definecolor{orchid1}{rgb}{1,0.51,0.98}
\definecolor{orchid2}{rgb}{0.93,0.48,0.91}
\definecolor{orchid3}{rgb}{0.8,0.41,0.79}
\definecolor{orchid4}{rgb}{0.54,0.28,0.54}
\definecolor{plum}{rgb}{0.86,0.63,0.86}
\definecolor{plum1}{rgb}{1,0.73,1}
\definecolor{plum2}{rgb}{0.93,0.68,0.93}
\definecolor{plum3}{rgb}{0.8,0.59,0.8}
\definecolor{plum4}{rgb}{0.54,0.4,0.54}
\definecolor{purple}{rgb}{0.63,0.13,0.94}
\definecolor{purple1}{rgb}{0.61,0.19,1}
\definecolor{purple2}{rgb}{0.57,0.17,0.93}
\definecolor{purple3}{rgb}{0.49,0.15,0.8}
\definecolor{purple4}{rgb}{0.33,0.1,0.54}
\definecolor{thistle}{rgb}{0.84,0.75,0.84}
\definecolor{thistle1}{rgb}{1,0.88,1}
\definecolor{thistle2}{rgb}{0.93,0.82,0.93}
\definecolor{thistle3}{rgb}{0.8,0.71,0.8}
\definecolor{thistle4}{rgb}{0.54,0.48,0.54}
\definecolor{violet}{rgb}{0.93,0.51,0.93}
\definecolor{AntiqueWhite}{rgb}{0.98,0.92,0.84}
\definecolor{AntiqueWhite1}{rgb}{1,0.93,0.86}
\definecolor{AntiqueWhite2}{rgb}{0.93,0.87,0.8}
\definecolor{AntiqueWhite3}{rgb}{0.8,0.75,0.69}
\definecolor{AntiqueWhite4}{rgb}{0.54,0.51,0.47}
\definecolor{FloralWhite}{rgb}{1,0.98,0.94}
\definecolor{GhostWhite}{rgb}{0.97,0.97,1}
\definecolor{NavajoWhite}{rgb}{1,0.87,0.68}
\definecolor{NavajoWhite1}{rgb}{1,0.87,0.68}
\definecolor{NavajoWhite2}{rgb}{0.93,0.81,0.63}
\definecolor{NavajoWhite3}{rgb}{0.8,0.7,0.54}
\definecolor{NavajoWhite4}{rgb}{0.54,0.47,0.37}
\definecolor{OldLace}{rgb}{0.99,0.96,0.9}
\definecolor{WhiteSmoke}{rgb}{0.96,0.96,0.96}
\definecolor{gainsboro}{rgb}{0.86,0.86,0.86}
\definecolor{ivory}{rgb}{1,1,0.94}
\definecolor{ivory1}{rgb}{1,1,0.94}
\definecolor{ivory2}{rgb}{0.93,0.93,0.88}
\definecolor{ivory3}{rgb}{0.8,0.8,0.75}
\definecolor{ivory4}{rgb}{0.54,0.54,0.51}
\definecolor{linen}{rgb}{0.98,0.94,0.9}
\definecolor{seashell}{rgb}{1,0.96,0.93}
\definecolor{seashell1}{rgb}{1,0.96,0.93}
\definecolor{seashell2}{rgb}{0.93,0.89,0.87}
\definecolor{seashell3}{rgb}{0.8,0.77,0.75}
\definecolor{seashell4}{rgb}{0.54,0.52,0.51}
\definecolor{snow}{rgb}{1,0.98,0.98}
\definecolor{snow1}{rgb}{1,0.98,0.98}
\definecolor{snow2}{rgb}{0.93,0.91,0.91}
\definecolor{snow3}{rgb}{0.8,0.79,0.79}
\definecolor{snow4}{rgb}{0.54,0.54,0.54}
\definecolor{wheat}{rgb}{0.96,0.87,0.7}
\definecolor{wheat1}{rgb}{1,0.9,0.73}
\definecolor{wheat2}{rgb}{0.93,0.84,0.68}
\definecolor{wheat3}{rgb}{0.8,0.73,0.59}
\definecolor{wheat4}{rgb}{0.54,0.49,0.4}
\definecolor{white}{rgb}{1,1,1}
\definecolor{BlanchedAlmond}{rgb}{1,0.92,0.8}
\definecolor{DarkGoldenrod}{rgb}{0.72,0.52,0.04}
\definecolor{DarkGoldenrod1}{rgb}{1,0.72,0.06}
\definecolor{DarkGoldenrod2}{rgb}{0.93,0.68,0.05}
\definecolor{DarkGoldenrod3}{rgb}{0.8,0.58,0.05}
\definecolor{DarkGoldenrod4}{rgb}{0.54,0.39,0.03}
\definecolor{LemonChiffon}{rgb}{1,0.98,0.8}
\definecolor{LemonChiffon1}{rgb}{1,0.98,0.8}
\definecolor{LemonChiffon2}{rgb}{0.93,0.91,0.75}
\definecolor{LemonChiffon3}{rgb}{0.8,0.79,0.64}
\definecolor{LemonChiffon4}{rgb}{0.54,0.54,0.44}
\definecolor{LightGoldenrod}{rgb}{0.93,0.86,0.51}
\definecolor{LightGoldenrod1}{rgb}{1,0.92,0.54}
\definecolor{LightGoldenrod2}{rgb}{0.93,0.86,0.51}
\definecolor{LightGoldenrod3}{rgb}{0.8,0.74,0.44}
\definecolor{LightGoldenrod4}{rgb}{0.54,0.5,0.3}
\definecolor{LightGoldenrodYellow}{rgb}{0.98,0.98,0.82}
\definecolor{LightYellow}{rgb}{1,1,0.88}
\definecolor{LightYellow1}{rgb}{1,1,0.88}
\definecolor{LightYellow2}{rgb}{0.93,0.93,0.82}
\definecolor{LightYellow3}{rgb}{0.8,0.8,0.7}
\definecolor{LightYellow4}{rgb}{0.54,0.54,0.48}
\definecolor{PaleGoldenrod}{rgb}{0.93,0.91,0.66}
\definecolor{PapayaWhip}{rgb}{1,0.93,0.83}
\definecolor{cornsilk}{rgb}{1,0.97,0.86}
\definecolor{cornsilk1}{rgb}{1,0.97,0.86}
\definecolor{cornsilk2}{rgb}{0.93,0.91,0.8}
\definecolor{cornsilk3}{rgb}{0.8,0.78,0.69}
\definecolor{cornsilk4}{rgb}{0.54,0.53,0.47}
\definecolor{gold}{rgb}{1,0.84,0}
\definecolor{gold1}{rgb}{1,0.84,0}
\definecolor{gold2}{rgb}{0.93,0.79,0}
\definecolor{gold3}{rgb}{0.8,0.68,0}
\definecolor{gold4}{rgb}{0.54,0.46,0}
\definecolor{goldenrod}{rgb}{0.85,0.64,0.13}
\definecolor{goldenrod1}{rgb}{1,0.75,0.14}
\definecolor{goldenrod2}{rgb}{0.93,0.7,0.13}
\definecolor{goldenrod3}{rgb}{0.8,0.61,0.11}
\definecolor{goldenrod4}{rgb}{0.54,0.41,0.08}
\definecolor{moccasin}{rgb}{1,0.89,0.71}
\definecolor{yellow}{rgb}{1,1,0}
\definecolor{yellow1}{rgb}{1,1,0}
\definecolor{yellow2}{rgb}{0.93,0.93,0}
\definecolor{yellow3}{rgb}{0.8,0.8,0}
\definecolor{yellow4}{rgb}{0.54,0.54,0}

\endinput
%%
%% End of file `BeamerColor.sty'.
\uselanguage{romanian}
\languagepath{romanian}

\deftranslation[to=romanian]{Proof}{Demonstra\c tie}
\deftranslation[to=romanian]{Example}{Exemplu}
\deftranslation[to=romanian]{Theorem}{Teorem\u a}
\deftranslation[to=romanian]{Solution}{Solu\c tie}
\deftranslation[to=romanian]{Lemma}{Lem\u a}
\deftranslation[to=romanian]{Definition}{Defini\c tie}

\usepackage{alltt}
\usepackage{xcolor}
\usepackage{float}
\usepackage{graphicx,wrapfig}
\usepackage{multirow}
\usepackage{tabularx,colortbl}
\usepackage{listings}  
\usepackage{multicol}  
\usepackage{hyperref}  
\usepackage{tikz}
 
\newcommand{\intens}[1] {{\color{DeepSkyBlue3} #1}}
\newcommand{\myalert}[1] {{\color{MedianOrange} #1}}

\usepackage{tslides}
\usepackage{comment}

\lstset{language=Haskell}
\lstset{escapeinside={(*@}{@*)}}
\newcommand{\li}[1]{\lstinline$#1$}
\newcommand{\ra}{\rightarrow}
\newcommand{\sra}{\stackrel{*}{\rightarrow}}


\newcommand{\SSnot}{\terminal{not}}

\newcommand{\Sand}{\terminal{and}}
\newcommand{\Sor}{\terminal{or}}
\newcommand{\Splus}{\terminal{+}}
\newcommand{\Smul}{\terminal{*}}
\newcommand{\Ssucc}{\terminal{S}}
\newcommand{\Spow}{\terminal{pow}}
\newcommand{\Spred}{\terminal{pred}}
\newcommand{\Seq}{\terminal{eq}}
\newcommand{\Sneq}{\terminal{neq}}

\newcommand{\SisZero}{\terminal{isZero}}
\newcommand{\Slte}{\terminal{<=}}
\newcommand{\Sgte}{\terminal{>=}}
\newcommand{\Slt}{\terminal{<}}
\newcommand{\Sgt}{\terminal{>}}
\newcommand{\Spair}{\terminal{pair}}
\newcommand{\Sfst}{\terminal{fst}}
\newcommand{\Ssnd}{\terminal{snd}}
\newcommand{\Sminus}{\terminal{-}}

\newcommand{\Snull}{\terminal{null}}
\newcommand{\Scons}{\terminal{cons}}
\newcommand{\Shead}{\terminal{head}}
\newcommand{\SisNull}{\terminal{?null=}}
\newcommand{\Stail}{\terminal{tail}}
\newcommand{\Ssum}{\terminal{sum}}
\newcommand{\Sfoldr}{\terminal{foldr}}
\newcommand{\Smap}{\terminal{map}}
\newcommand{\Sfilter}{\terminal{filter}}
 
\begin{document}
\title{\\Curs 7}
\author{Fundamentele Limbajelor de Programare} 
\date{2020-2021} 

\frame{\titlepage} 

 

%\frame{\frametitle{Cuprins}\tableofcontents} 

\begin{frame}{$\lambda$-calcul și calculabilitate}

  \begin{itemize}
  \item În 1929-1932 Church a propus 
  $\lambda$-calculul ca sistem formal pentru logica matematică.
  În 1935 a argumentat că orice funcție calculabilă peste numere naturale poate
  fi calculată in $\lambda$-calcul.
  
  \item În 1935, independent de Church, Turing a dezvoltat mecanismul de calcul
  numit astăzi Mașina Turing. 
  În 1936 și el a argumentat câ orice funcție calculabilă peste numere naturale poate
  fi calculată de o mașină Turing.
  De asemenea, a arătat echivalența celor două modele de calcul.
  Această echivalență a constituit o indicație puternică asupra "universalității" 
  celor două modele, conducând la ceea ce numim astăzi "Teza Church-Turing".
  \end{itemize}
\end{frame}


\begin{frame}[fragile]{ $\lambda$-calcul: sintaxa}

  \begin{center}
   \begin{tabular}{lll}
  $t=$ &  $x$ &  (variabilă) \\
   & $\mid (\lambda x.\, t)$ & (abstractizare)\\
    & $\mid (t\; t)$ & (aplicare)
  \end{tabular}
  \end{center}

Conven\ts ii:

\begin{itemize}
\item se elimin\u a parantezele exterioare
\item aplicarea este asociativ\u a la st\^{\i}nga: $t_1t_2t_3$ este $(t_1t_2)t_3$
\item corpul abstractiz\u arii este extins la dreapta: $\lambda x.t_1t_2$ este $\lambda x.(t_1t_2)$ (nu $(\lambda x.t_1)t_2$)
\item scriem $\lambda xyz.t$ \^{\i}n loc de $\lambda x.\lambda y.\lambda z.t$
\end{itemize}

\pause

\begin{block}{Întrebare}
  Ce putem exprima / calcula folosind {\bf doar} $\lambda$-calcul?
\end{block}

\end{frame}  

\section{Expresivitatea $\lambda$-calculului}

\begin{frame}{Rezumat}
  \begin{itemize}
    \item Reprezentarea valorilor de adevăr și a expresiilor condiționale
    \item Reprezentarea perechilor (tuplurilor) și a funcțiilor proiecție
    \item Reprezentarea numerelor și a operatiilor aritmetice de bază
    \item Recursie
  \end{itemize}
\end{frame}

\begin{frame}{Ideea generală}
  \begin{block}{Intuiție}
    Tipurile de date sunt codificate de {\em capabilități}
  \end{block}

  \begin{description}
    \item[Boole] capabilitatea de a alege între două alternative
    \item[Perechi] capabilitatea de a calcula ceva bazat pe două valori
    \item[Numere naturale] capabilitatea de a itera de un număr dat de ori
  \end{description}
\end{frame}

\subsection{Valori de adevăr}

\begin{frame}{Valori de adevăr}

\begin{description}
  \item[Intuiție:] Capabilitatea de a alege între două alternative.
  \item[Codificare:] Un Boolean este o funcție cu 2 argumente
       
        reprezentând ramurile unei alegeri.
  \item[$\Strue$ ::=] $\lambda t\; f. t$
      --- din cele două alternative o alege pe prima
  \item[$\Sfalse$ ::=] $\lambda t\; f. f$
      --- din cele două alternative o alege pe a doua
\end{description}

\end{frame}

\begin{frame}{Operații Booleene}
  \begin{description}
  \item[$\Strue$ ::=] $\lambda t\; f. t$
      --- din cele două alternative o alege pe prima
  \item[$\Sfalse$ ::=] $\lambda t\; f. f$
      --- din cele două alternative o alege pe a doua
  \item[$\Sif$ ::= ] $\lambda c\; then\; else. c\; then\; else$
  --- pur și simplu folosim valoarea de adevăr pentru a alege între alternative

  $\Sif \Sfalse\; (\lambda x.x\; x)\; (\lambda x.x) \pause \rightarrow^3_\beta
   \Sfalse\; (\lambda x.x\; x)\; (\lambda x.x) \rightarrow^2_\beta \lambda x.x$

  \item[$\Sand$ ::= ] $\lambda b1\; b2.\Sif b1\; b2\; \Sfalse$ sau  $\lambda b1\; b2.b1\; b2\; b1$

  $\Sand \Strue\; \Sfalse \pause \rightarrow^2_\beta \Strue\; \Sfalse\; \Strue \rightarrow^2_\beta \Sfalse$
  \item[$\Sor$ ::= ] $\lambda b1\; b2.\Sif b1\; \Strue\; b2$ sau  $\lambda b1\; b2.b1\; b1\; b2$

  $\Sor \Strue\; \Sfalse \pause \rightarrow^2_\beta \Strue\; \Strue\; \Sfalse \rightarrow^2_\beta \Strue$
  \item[$\SSnot$ ::= ] $\lambda b.\Sif b\; \Sfalse\; \Strue$ sau $\lambda b\; t\; f.b\; f\; t$
  
  $\SSnot \Strue \pause \rightarrow_\beta \lambda t\; f.\Strue\; f\; t \rightarrow_\beta \lambda t\; f.f$
  \end{description}
\end{frame}

\subsection{Numere naturale}

\begin{frame}{Numere naturale}
  \begin{description}
  \item[Intuiție:] Capabilitatea de a itera o funcție de un număr de ori peste o valoare inițială
  \item[Codificare:] Un număr natural este o funcție cu 2 argumente
       
        \begin{itemize}
          \item[s] funcția care se iterează
          \item[z] valoarea inițială
        \end{itemize}
  \item[0 ::=] $\lambda s\; z. z$
      --- $s$ se iterează de 0 ori, deci valoarea inițială
  \item<2->[1 ::=] $\lambda s\; z. s\; z$
      --- funcția iterată o dată aplicată valorii inițiale
  \item<3->[2 ::=] $\lambda s\; z. s (s\; z)$
      --- $s$ iterată de 2 ori, aplicată valorii inițiale
  \item<4->[...]
  \item<4->[8 ::=] $\lambda s\; z. s (s (s (s (s (s (s (s\; z)))))))$
  \item<4->[...]
  \item[]
  \item<5->[Observație:] $0 = \Sfalse$
  \end{description}

\end{frame}

\begin{frame}{Operații aritmetice de bază}
\begin{description}
  \item[0 ::=] $\lambda s\; z. z$
      --- $s$ se iterează de 0 ori, deci valoarea inițială
  \item[8 ::=] $\lambda s\; z. s (s (s (s (s (s (s (s\; z)))))))$
  \item[$\Ssucc$ ::=] $\lambda n\; s\; z.s\; (n\; s\; z)$ sau  $\lambda n\; s\; z.n\; s\; (s z)$
    
  $\Ssucc 0 \pause \rightarrow_\beta \lambda s\; z.0 s (s z) \rightarrow^2_\beta
    \lambda s\; z. s z
    = 1
  $
  \item[$\Splus$ ::=] $\lambda m\; n. m\; \Ssucc\; n$ sau $\lambda m\; n.\lambda s\; z.m\; s\; (n\; s\; z)$

  $\Splus 3\; 2 \pause \rightarrow^2_\beta \lambda s\; z.3\; s\; (2\; s\; z) \rightarrow^2_\beta \lambda s\; z. s (s (s (2\; s\; z))) \rightarrow^2_\beta
    \lambda s\; z. s (s (s (s (s\; z)))
    = 5
    $
  \item[$\Smul$ ::=] $\lambda m\; n. m\; (\Splus n)\; 0$ sau $\lambda m\; n.\lambda s.m\; (n\; s)$

  $\Smul 3\; 2 \pause \rightarrow^2_\beta 3\; (\Splus 2) 0 \rightarrow^2_\beta
    \Splus 2 (\Splus 2 (\Splus 2\; 0)) \rightarrow^4_\beta$ 
    $\Splus 2 (\Splus 2\; 2)  \rightarrow^4_\beta
     \Splus 2\; 4  \rightarrow^4_\beta 6$

  \item[$\Spow$ ::=] $\lambda m\; n. n\; (\Smul m)\; 1$ sau $\lambda m\; n.n\; m$

  $\Spow 3\; 2 \pause \rightarrow^2_\beta 2\; 3 \rightarrow^2_\beta
    \lambda z.3 (3\; z) \rightarrow_\beta
     \lambda z.\lambda z'.3\; z (3\; z (3\; z\; z')) \equiv_\alpha
     \lambda s\; z.3\; s (3\; s (3\; s\; z)) \rightarrow^6_\beta
     \lambda s\; z.s ( s ( s (s ( s ( s (s (s (s \; z))))))))
     = 9
  $
\end{description}
\end{frame}

\begin{frame}{Scăderi și comparații}
  Presupunem că avem o funcție predecesor
  $\Spred x = \left\{\begin{array}{lr}0 & x = 0\\x-1 & x\neq 0 \end{array}\right.$
  \begin{description}
  \item[$\Sminus$ ::= ] \pause $\lambda m\; n.n \Spred m$  --- dă 0 dacă $m \leq n$
  \item[$\SisZero$ ::= ] \pause $\lambda n.n (\lambda x.\Sfalse) \Strue$ --- testează dacă $n$ e $0$
  \item[$\Slte$ ::=] \pause $\lambda m\; n.\SisZero (\Sminus m\; n)$
  \item[$\Sgt$ ::=] \pause $\lambda m\; n.\SSnot (\Slte m\; n)$
  \item[$\Sgte$ ::=] $\lambda m\; n.\Slte n\; m$
  \item[$\Slt$ ::=] $\lambda m\; n.\Sgt n\; m$
  \item[$\Seq$ ::=] \pause $\lambda m\; n.\Sand (\Slte m\; n)\; (\Sgte m\; n)$
  \item[$\Sneq$ ::=] $\lambda m\; n.\SSnot (\Seq m\; n)$
  \end{description}

  \begin{block}{Problemă}
    Cum definim funcția $\Spred$?
  \end{block}
\end{frame}

\begin{frame}{Funcția predecesor
  $\Spred x = \left\{\begin{array}{lr}0 & x = 0\\x-1 & x\neq 0 \end{array}\right.$}

  \begin{description}
    \item[Idee 1:] $\Spred$ ::= $\lambda n.\SisZero n\; 0\;  (\Spred' n)$
           --- am tratat primul caz. acum vrem o funcție $\Spred'$ care calculeaza $n-1$ dacă $n\neq 0$
    \item[Idee 2:] folosim iterația $\Spred'$ ::= $\lambda n. n \Ssucc' Z'$, unde
         \begin{itemize}
          \item $\Ssucc'$ e un fel de succesor
          \item $Z'$ e un fel de $-1$, i.e. $\Ssucc' Z' = 0$
         \end{itemize}
    \item[$\Ssucc'$ ::= ] $\lambda n. n \Ssucc 1$
    \item[$Z'$ ::= ] $\lambda s\; z.0$ --- $Z'$ nu e codificarea unui număr

    Dar se verifică că $\Ssucc' Z' \rightarrow_\beta Z' \Ssucc 1 \rightarrow^2_\beta 0$

    \item[Totul e OK] deoarece $\Spred'$ e aplicată {\bf doar} pe numere diferite de 0.
    \item[$\Spred$ ::= ] $\lambda n.\SisZero n\; 0\; (n \; (\lambda n.n\; \Ssucc 1)\; (\lambda s\; z.0))$
  \end{description}
\end{frame}

\section{Perechi (și tupluri)}

\begin{frame}{Perechi}

  \begin{description}
    \item[Intuiție:] Capabilitatea de a aplica o funcție componentelor perechii
    \item[Codificare:] O funcție cu 3 argumente 
         
          reprezentând componentele perechii și funcția ce vrem să o aplicăm lor.
    \item[$\Spair$ ::=] $\lambda x\; y.\lambda f.f\; x\; y$

    Constructorul de perechi

  \end{description}
    \begin{block}{Exemplu: $\Spair 3\; 5 \rightarrow^2_\beta \lambda f.f\; 3\; 5$}
    
    perechea $(3,5)$
    reprezintă capabilitatea de a aplica o funcție de două argumente lui $3$ si apoi lui $5$.
    \end{block}
\end{frame}

\begin{frame}{Operații pe perechi}

  \begin{description}
    \item[$\Spair$ ::=] $\lambda x\; y.\lambda f.f\; x\; y$
    \item[$\Spair x y \equiv_\beta$] $f\; x\; y$
    \item[]
    \item[$\Sfst$ ::=] $\lambda p.p\; \Strue$ --- $\Strue$ alege prima componentă

    $\Sfst (\Spair 3\; 5) \rightarrow_\beta
    {\Spair 3\; 5\; \Strue} \rightarrow^3_\beta
    \Strue\; 3\; 5 \rightarrow^2_\beta 3$

    \item[$\Ssnd$ ::=] $\lambda p.p\; \Sfalse$ --- $\Sfalse$ alege a doua componentă

    $\Ssnd (\Spair 3\; 5) \rightarrow_\beta
    {\Spair 3\; 5\; \Sfalse} \rightarrow^3_\beta
    \Sfalse\; 3\; 5 \rightarrow^2_\beta 5$
  \end{description}
\end{frame}

\begin{frame}{Definirea funcției $\Spred$ folosind perechi}

  $$\Spred x = \left\{\begin{array}{lr}0 & x = 0\\x-1 & x\neq 0 \end{array}\right.$$

  \pause

  $$\Spred'' = \lambda n. n \Ssucc'' (\Spair 0\; 0)$$

  \pause

  $$\Ssucc'' = \lambda p.(\lambda x.\Spair x (\Ssucc x))\; (\Ssnd p)$$

  \pause

  $$\Spred = \lambda n.\Sfst(\Spred'' n)$$

  \pause

  $\Spred 2 \rightarrow_\beta {\Sfst (\Spred'' 2)} \rightarrow_\beta
  {\Sfst (2 \Ssucc'' (\Spair 0\; 0))} \rightarrow^2_\beta
  {\Sfst (\Ssucc'' (\Ssucc'' (\Spair 0\; 0)))} \rightarrow_\beta
  {\Sfst (\Ssucc'' (\Ssucc'' (\Spair 0\; 0)))} \rightarrow_\beta
  {\Sfst (\Ssucc'' ((\lambda x.\Spair x (\Ssucc x))\; (\Ssnd (\Spair 0\; 0))))} \rightarrow^6_\beta
  {\Sfst (\Ssucc'' ((\lambda x.\Spair x (\Ssucc x))\; 0))} \rightarrow_\beta
  {\Sfst (\Ssucc'' (\Spair 0 (\Ssucc 0)))} \rightarrow^8_\beta
  {\Sfst (\Spair (\Ssucc 0) (\Ssucc (\Ssucc 0)))} \rightarrow^6_\beta
  {\Ssucc 0} \rightarrow^3_\beta
  1
  $
\end{frame}

\begin{frame}{Factorial, Fibonacci, împărțire folosind perechi}
  \begin{description}
    \item[$\terminal{fact}$ ::= ] $\lambda n.\Ssnd (n\;(\lambda p. \Spair (\Ssucc (\Sfst p)) (\Smul (\Sfst p) (\Ssnd p)))\;(\Spair 1\; 1))$
    \pause
    \item[$\terminal{fib}$ ::= ] $\lambda n.\Sfst (n\; (\lambda p. \Spair (\Ssnd p) (\Splus (\Sfst p)\; (\Ssnd p)))\; (\Spair 0\; 1))$
    \pause
    \item[$\terminal{divMod}$ ::= ] $\lambda m\; n. m\; (\lambda p. \Sgt n\; (\Ssnd p)\; p\; (\Spair (\Ssucc (\Sfst p))\; (\Sminus (\Ssnd p)\; n)))\; (\Spair 0\; m)$
  \end{description}

    \pause
  \begin{block}{Observații}
    \begin{itemize}
      \item Nu toate funcțiile pot fi definite prin iterare fixată
      \item Pentru $\terminal{divMod}$ obținem răspunsul (de obicei) din mult mai puțin de $m$ iterații
    \end{itemize}
  \end{block}
\end{frame}

\section{Recursie}
\begin{frame}[fragile]{Recursie}
\begin{itemize}
\item exist\u a termeni care {\bf nu} pot fi redu\sh i la o $\beta$-form\u a normal\u a, de exemplu 

$(\lambda x. xx)(\lambda x. xx)\ra_\beta (\lambda x. xx)(\lambda x. xx)$

\^{I}n $\lambda$-calcul putem defini calcule infinite!

\pause

Dac\u a not\u am $Af :: = \lambda x. f (xx)$ atunci 

$(Af)(Af)=_\beta (\lambda x. f(xx))(Af)=_\beta f((Af)(Af))$

Dac\u a not\u am $Yf :: = (Af)(Af)$ atunci $Yf =_\beta f(Yf)$.

\end{itemize}
\end{frame}

\begin{frame}[fragile]{Puncte fixe}

\begin{itemize}
\item pentru o func\ts ie $f:X\to X$ un {\bf punct fix} este un element $x_0\in X$ cu $f(x_0)=x_0$.


\begin{itemize}
\item $f:\Nat\to\Nat f(x)=x+1$ nu are puncte fixe
\item $f:\Nat\to\Nat f(x)=2x$ are punctul fix $x=0$
\end{itemize} 

\pause

\item \^{I}n $\lambda$-calcul  
  \begin{description}
    \item[$\terminal{Y}$ ::= ] $\lambda f. (\lambda x. f(xx))(\lambda x. f(xx))$
\end{description}
    
are proprietatea c\u a $Yf =_\beta f(Yf)$, deci $Yf$ este un {\bf punct fix} pentru $f$.     
    \pause

\medskip

$Y$ se nume\sh te {\bf combinator de punct fix}. \pause\medskip


Avem $Yf =_\beta f(Yf) =_\beta f(f(YF)) =_\beta \ldots$

\medskip
Putem folosi $Y$ pentru a ob\ts ine apeluri recursive!
\end{itemize}
\end{frame}

\begin{frame}[fragile]{Puncte fixe - func\ts ia factorial}

  \begin{description}
    \item[$\terminal{fact}$ ::=] $\lambda n. \,\,if\,\, (\terminal{isZero}\,\, n)\,\, \terminal{one}\,\, (*\, n\, {\textcolor{red}{fact}} (\terminal{pred} \,\,n))$
    \end{description}
    \pause\medskip
Aceast\u a defini\ts ie nu este corect\u a conform regulilor noastre, cum proced\u am?\pause\medskip
\begin{itemize}
\item Pasul 1: abstractiz\u am, astfel \^{\i}nc\^ at construc\ts ia s\u a fie corect\u a

\begin{description}
    \item[$\terminal{factA}$ ::=] $\lambda f\lambda n. \,\,if\,\, (\terminal{isZero}\,\, n)\,\, \terminal{one}\,\, (*\, n\, ( f (\terminal{pred} \,\,n)))$
    \end{description}
\pause
\item Pasul 2: aplic\u am combinatorul de punct fix

\begin{description}
    \item[$\terminal{fact}$ ::=] $ Y \, \terminal{factA}$
    \end{description}
    \pause
    
Deoarece $Y\,\terminal{factA} =_\beta \terminal{factA}(Y\,\terminal{factA})$ ob\ts inem 

   $$\terminal{fact} =_\beta \lambda n. \,\,if\,\, (\terminal{isZero}\,\, n)\,\, \terminal{one}\,\, (*\, n\, (\terminal{fact} (\terminal{pred} \,\,n)))$$ 
\end{itemize}    
\end{frame}

\begin{frame}[fragile]{Puncte fixe - func\ts ia factorial}

$\begin{array}{l}
\terminal{fact}\,\,  \terminal{zero}  \,\,=_\beta \,\,(Y\, \terminal{factA}) \terminal{zero}\\
 =_\beta \,\,\terminal{factA} (Y\, \terminal{factA}) \terminal{zero}\\
 =_\beta \,\, if\,\, (\terminal{isZero}\,\terminal{zero})\,\, \terminal{one}\,\, (*\, \terminal{zero}\, ((Y\,\terminal{factA}) (\terminal{pred} \,\,\terminal{zero})))\\
 =_\beta \,\,\terminal{one}
\end{array}$    \pause
\medskip

$\begin{array}{l}
\terminal{fact}\,\,  \terminal{one}  \,\,=_\beta \,\,(Y\, \terminal{factA}) \terminal{one}\\
 =_\beta \,\,\terminal{factA} (Y\, \terminal{factA}) \terminal{one}\\
 =_\beta \,\, if\,\, (\terminal{isZero}\,\terminal{one})\,\, \terminal{one}\,\, (*\, \terminal{one}\, ((Y\,\terminal{factA}) (\terminal{pred} \,\,\terminal{one})))\\
 =_\beta \,\, *\, \terminal{one}\, ((Y\,\terminal{factA}) (\terminal{pred} \,\,\terminal{one}))\\
 =_\beta \ldots
\end{array}$
\end{frame}

\section{Liste}
\begin{frame}{Liste}

  \begin{description}
    \item[Intuiție:] Capabilitatea de a agrega o listă
    \item[Codificare:] O funcție cu 2 argumente 
         
        funcția de agregare și valoarea inițială
    \item[$\Snull$ ::=] $\lambda a\; i.i$ --- lista vidă
    \item[$\Scons$ ::=] $\lambda x\; l.\lambda a\; i.a\; x\; (l\; a\; i)$

    Constructorul de liste

  \end{description}
    \begin{block}{Exemplu: $\Scons 3\; (\Scons 5 \Snull) \rightarrow^2_\beta
      \lambda a\; i.a\; 3\; (\Scons 5 \Snull a\; i) \rightarrow^4_\beta
      \lambda a\; i.a\; 3\; (a\; 5\; (\Snull a\; i)) \rightarrow^2_\beta
      \lambda a\; i.a\; 3\; (a\; 5\; i)
      $}
    
    Lista $[3, 5]$
    reprezintă capabilitatea de a agrega elementele $3$ si apoi $5$ dată
    fiind o funcție de agregare $a$ și o valoare implicită $i$.

    Pentru aceste liste, operația de bază este \texttt{foldr}.
    \end{block}
\end{frame}

\begin{frame}{Operații pe liste}

  \begin{description}
    \item[$\Snull$ ::=] $\lambda a\; i.i$ --- lista vidă
    \item[$\Scons$ ::=] $\lambda x\; l.\lambda a\; i.a\; x\; (l\; a\; i)$
    \item[]
    \pause
    \item[$\SisNull$ ::=] $\lambda l. l\; (\lambda x\; v.\Sfalse)\; \Strue$
    \pause
    \item[$\Shead$ ::=] $\lambda d\; l. l\; (\lambda x\; v.x)\; d$

    primul element al listei, sau $d$ dacă lista e vidă

    \pause
    \item[$\Stail$ ::=] $\lambda l. \Sfst (l\; (\lambda x\; p.\Spair (\Ssnd p)\; (\Scons x\; (\Ssnd p)))\; (\Spair\; \Snull\; \Snull))$

    coada listei, sau lista vidă dacă lista e vidă

    \item[$\Sfoldr$ ::=] $\lambda f\; i\; l. l\; f\; i$
    \item[$\Smap$ ::=] $\lambda f\; l. l\; (\lambda x\; t.\Scons (f\; x)\; t)\; \Snull$
    \item[$\Sfilter$ ::=] $\lambda p\; l. l\; (\lambda x\; t.p\; x\; (\Scons x\; t)\; t)\; \Snull$

  \end{description}
\end{frame}

\begin{frame}{}
\vfill\begin{center}
\intens{Pe s\u apt\u am\^ana viitoare!}
\end{center}
\vfill
\end{frame}
\end{document}