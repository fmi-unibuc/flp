\documentclass[xcolor=pdftex,romanian,colorlinks]{beamer}
%\documentclass[xcolor=pdftex,handout,romanian,colorlinks]{beamer}

\usepackage[export]{adjustbox}
\usepackage{../sty/tslides}
\usepackage[all]{xy}
\usepackage{pgfplots}
\usepackage{flowchart}
\usepackage{todonotes}
\usepackage{multicol}  
\usetikzlibrary{arrows,positioning,calc}
\lstset{language=Haskell}
\lstset{escapeinside={(*@}{@*)}}
\PrerenderUnicode{ăĂîÎȘșȚțâÂ}
\usepackage{amsmath}


%\usepackage{xcolor}
%\definecolor{IntensColor}{HTML}{2E86C1}
%\definecolor{StateTransition}{HTML}{D6EAF8}
%\definecolor{MedianLightOrange}{RGB}{216,178,92}
%\definecolor{Orchid}{HTML}{8E44AD}
%\definecolor{True}{HTML}{229954}
%\definecolor{False}{HTML}{CB4335}


\usepackage{proof}
\usepackage{multirow}
\usepackage{alltt}
\usepackage{mathpartir}
\usepackage{ulem}

\newcommand{\structured}[1]{#1}

\definecolor{IntensColor}{HTML}{2E86C1}
\definecolor{StateTransition}{HTML}{D6EAF8}
\definecolor{MedianLightOrange}{RGB}{216,178,92}
\definecolor{Orchid}{HTML}{8E44AD}
\definecolor{True}{HTML}{229954}
\definecolor{False}{HTML}{CB4335}

\newcommand{\cin}[1]{{\color{cobalt} #1}}
\newcommand{\sel}[1]{{\color{Orchid} #1}}

\newcommand{\intens}[1] {{\color{IntensColor} #1}}
\newcommand{\exe}[1] {{\color{True} #1}}

\newcommand{\la}{\lambda}

\setlength{\leftmargini}{0pt}

\newcommand{\app}[2]{#1\, #2}
\newcommand{\abs}[2]{\lambda #1.\,#2}

\newcommand{\type}[2]{{\color{True}#1\hspace{-.05cm}:}\,{\color{Orchid}#2}}

\newcommand{\sub}[3]{#1\langle#2/#3\rangle}
\newcommand{\subt}[3]{#1[#2/#3]}
\newcommand{\equiva}{=_\alpha}

\newcommand{\trueL}{\mathbf{T}}
\newcommand{\falseL}{\mathbf{F}}
\newcommand{\notL}{\mathbf{not}}
\newcommand{\andL}{\mathbf{and}}
\newcommand{\orL}{\mathbf{or}}
\newcommand{\ifL}{\mathbf{if}}
\newcommand{\boolL}{\mathbf{bool}}

\newcommand{\maybeL}{\mathbf{maybe}}
\newcommand{\nothingL}{\mathbf{Nothing}}
\newcommand{\justL}{\mathbf{Just}}
\newcommand{\Maybe}[1]{\mathop{\mathbf{Maybe}}{#1}}

\newcommand{\foldrL}{\mathbf{foldr}}
\newcommand{\nilL}{\mathbf{Nil}}
\newcommand{\consL}{\mathbf{Cons}}
\newcommand{\ListL}[1]{\mathop{\mathbf{List}}{#1}}

\newcommand{\unpairL}{\mathbf{uncons}}
\newcommand{\pairL}{\mathbf{Pair}}
\newcommand{\firstL}{\mathbf{first}}
\newcommand{\secondL}{\mathbf{second}}
\newcommand{\Pair}[2]{\mathop{\mathop{\mathbf{Pair}}{#1}}{#2}}

\newcommand{\succL}{\mathbf{Succ}}
\newcommand{\zeroL}{\mathbf{Zero}}
\newcommand{\iterateL}{\mathbf{iterate}}
\newcommand{\addL}{\mathbf{add}}
\newcommand{\mulL}{\mathbf{mul}}
\newcommand{\expL}{\mathbf{exp}}
\newcommand{\isZero}{\mathbf{isZero}}
\newcommand{\pred}{\mathbf{pred}}
\newcommand{\factL}{\mathbf{fact}}

\newcommand{\BoolT}{\ensuremath{\texttt{Bool}}}
%\newcommand{\BoolT}{\ensuremath{\texttt{Bool}}}
\newcommand{\ifT}[3]{\mathrm{if}\ #1\ \mathrm{then}\ #2\ \mathrm{else}\ #3}

\newcommand{\UnitT}{\ensuremath{\texttt{Unit}}}
\newcommand{\unit}{\mathrm{unit}}

\newcommand{\VoidT}{\ensuremath{\texttt{Void}}}
\newcommand{\void}{\mathrm{void}}


\newcommand{\ProductT}[2]{#1 \times #2}
\newcommand{\PairL}[2]{\langle #1,#2\rangle}
\newcommand{\ProjOne}[1]{fst\ #1}
\newcommand{\ProjTwo}[1]{snd\ #1}

\newcommand{\SumT}[2]{#1 + #2}
\newcommand{\Left}[1]{\mathrm{Left}\ #1}
\newcommand{\Right}[1]{\mathrm{Right}\ #1}
\newcommand{\Case}[3]{\mathrm{case}\ #1\ \mathrm{of}\ #2\ ;\ #3}

%----------------------------------------------

\newcommand{\SSnot}{\terminal{not}}

\newcommand{\Sand}{\terminal{and}}
\newcommand{\Sor}{\terminal{or}}
\newcommand{\Splus}{\terminal{+}}
\newcommand{\Smul}{\terminal{*}}
\newcommand{\Ssucc}{\terminal{S}}
\newcommand{\Spow}{\terminal{pow}}
\newcommand{\Spred}{\terminal{pred}}
\newcommand{\Seq}{\terminal{eq}}
\newcommand{\Sneq}{\terminal{neq}}

\newcommand{\SisZero}{\terminal{isZero}}
\newcommand{\Slte}{\terminal{<=}}
\newcommand{\Sgte}{\terminal{>=}}
\newcommand{\Slt}{\terminal{<}}
\newcommand{\Sgt}{\terminal{>}}
\newcommand{\Spair}{\terminal{pair}}
\newcommand{\Sfst}{\terminal{fst}}
\newcommand{\Ssnd}{\terminal{snd}}
\newcommand{\Sminus}{\terminal{-}}

\newcommand{\Snull}{\terminal{null}}
\newcommand{\Scons}{\terminal{cons}}
%\newcommand{\c sead}{\terminal{head}}
\newcommand{\SisNull}{\terminal{?null}}
\newcommand{\Stail}{\terminal{tail}}
\newcommand{\Ssum}{\terminal{sum}}
\newcommand{\Sfoldr}{\terminal{foldr}}
\newcommand{\Smap}{\terminal{map}}
\newcommand{\Sfilter}{\terminal{filter}}

\newcommand{\const}[1]{\triangleright {\color{False} #1}}

\newcommand{\egf}[1]{\stackrel{\cdot}{=}_{#1}}

\newcommand{\Conf}[2]{\ensuremath{\langle #1\ ,\ #2\rangle}}
\newcommand{\plus}[1] {{\color{True} #1}}
\newcommand{\te}[1]{\mbox{\texttt{#1}}}

\newcommand{\vexp}{\ensuremath{\mathbb{E}}}
\newcommand{\bexp}{\ensuremath{\mathbb{B}}}
\newcommand{\cmd}{\ensuremath{\mathbb{C}}}

\definecolor{section-color}{HTML}{23373b} %mDarkTeal
%\AtBeginSection[]{
%  \begin{frame}
%  \vfill
%  \centering
%  \begin{beamercolorbox}[sep=8pt,center,shadow=true,rounded=true]{title}
%    \usebeamerfont{title}\insertsectionhead\par%
%  \end{beamercolorbox}
%  \vfill
%  \end{frame}
%}



\title[FLP]{Fundamentele limbajelor de programare}
\subtitle{C10}
\date{}


\begin{document}
\begin{frame}
  \titlepage
\end{frame}

\setlength{\leftmargini}{12pt}




%=========================================
\section{\color{section-color}Programare Logică \\ Logica Clauzelor Horn}  
%=========================================

\begin{frame}[fragile]{Program  în Prolog = mulțime de predicate}

Un exemplu de \alert{program în Prolog} din cursul trecut:
 
\begin{columns}
\begin{column}{.5\textwidth}
\vspace{-.2cm}
\begin{verbatim}
father(peter,meg). 
father(peter,stewie).

mother(lois,meg). 
mother(lois,stewie).

griffin(peter).
griffin(lois).

griffin(X) :- father(Y,X),  griffin(Y).
\end{verbatim}
\end{column}
\begin{column}{.3\textwidth}
\intens{\textbf{Predicate:}}\\
\intens{father/2}\\
\intens{mother/2}\\
\intens{griffin/1}
\end{column}
\end{columns}

\end{frame}

%---------------------------------------------------------------------

%---------------------------------------------------------------------
\begin{frame}{Spre logica din spatele Prologului}

Pentru a putea modela universul programului, fixăm un \intens{alfabet/signatură/vocabular}:
\begin{itemize}
	\item o mulțime $\mathbf{R}$ de \intens{simboluri de relații/predicate}
	\item o mulțime $\mathbf{F}$ de \intens{simboluri de operații/funcții}
	\item o funcție \intens{aritate} $ar: \mathbf{F} \cup \mathbf{R} \to \mathbb{N}$
	\item Fie $\mathbf{C}\subseteq \mathbf{F}$ mulțimea simbolurilor de operații
	de aritate $0$, numite și \intens{simboluri de constante}.
\end{itemize}

Notăm cu $\mathbf{R}_n$/$\mathbf{F}_n$ mulțimea simbolurilor de relații/operații
de aritate $n$. Observație: $\mathbf{F}_0 = C$

\textbf{\color{True} Exemplu.}
\begin{itemize}
	\item $\mathbf{R} = \mathbf{R}_1 \uplus \mathbf{R}_2$, unde $\mathbf{R}_1 = \{P\}$ și $\mathbf{R}_2 = \{R\}$
	\item $\mathbf{F} = \mathbf{F}_0 \uplus \mathbf{F}_2$, unde $\mathbf{F}_0 = C = \{c\}$ și $\mathbf{F}_2 = \{f\}$
\end{itemize}


\end{frame}
%---------------------------------------------------------------------

%---------------------------------------------------------------------
\begin{frame}{Sintaxa Prolog}

\begin{itemize}
	\item {Sintaxa Prolog nu face diferență între \intens{simboluri de operații} și \intens{simboluri de predicate}!}
	\vfill
	\item Dar este important când ne uităm la teoria corespunzătoare programului în logică să facem această distincție.
	\vfill
    \item \^{I}n sintaxa Prolog
    \begin{itemize}
    \item termenii compuși sunt predicate: \texttt{father(peter, meg)}
    \item operatorii sunt funcții: \texttt{+, *, mod}
    \end{itemize}
\end{itemize}
\end{frame}
%---------------------------------------------------------------------

%---------------------------------------------------------------------
\begin{frame}{Termeni}

Fixăm o mulțime numărabilă de \intens{variabile} $V$.

Definim \intens{termenii} inductiv astfel:
\begin{itemize}
	\item orice variabilă este un termen
	\item orice simbol de constantă este un termen
	\item dacă $f\in \mathbf{F}_n$, $n > 0$ și $t_1,\ldots,t_n$ sunt termeni, atunci $f(t_1,\ldots,t_n)$ este termen.
\end{itemize}

%\medskip
%Notăm cu \intens{$Trm_{\mathcal{L}}$} mulțimea termenilor lui $\mathcal{L}$.

\medskip
\textbf{\color{True}Exemple:} $c,\ x_1,\ f(x_1,c),\ f(f(x_2,x_2),c)$ \\
unde $c \in \mathbf{C}$ reprezintă o constantă, $x_1,x_2 \in V$ sunt variabile, iar $f \in \mathbf{F}_2$ reprezintă un simbol de funcție de 
aritate 2. 

\end{frame}
%---------------------------------------------------------------------

%---------------------------------------------------------------------
\begin{frame}{Formule atomice}

\intens{Formulele atomice} sunt definite astfel:
\vspace{-1cm}
\begin{center}
	\item dacă $R\in \mathbf{R}_0$, atunci $R$ este formulă atomică
	\item dacă $R\in \mathbf{R}_n$, $n > 0$ și $t_1,\ldots,t_n$ sunt termeni, \\ atunci \intens{$R(t_1,\ldots,t_n)$} este formulă atomică.
\end{center}

\medskip 
\textbf{\color{True}Exemple:} $P(f(x_1,c)),\ R(c,x_2)$ \\
unde $c \in \mathbf C$, $x_1,x_2 \in V$, $f \in \mathbf{F}_2$, $P \in \mathbf{R}_1$, iar $R \in \mathbf{R}_2$.
\end{frame}
%---------------------------------------------------------------------

%---------------------------------------------------------------------
\begin{frame}{Formulele logicii de ordinul I}

Definim \intens{formulele} astfel:
\begin{itemize}
	\item orice \intens{formulă atomică} este o formulă
	\item dacă $\varphi$ este o formulă, atunci \intens{$\neg \varphi$} este o formulă
	\item dacă $\varphi$ și $\psi$ sunt formule, atunci \intens{$\varphi \vee \psi$}, \intens{$\varphi \wedge \psi$}, \intens{$\varphi \to \psi$} sunt formule
	\item dacă $\varphi$ este o formulă și $x$ este o variabilă, atunci \intens{$\forall x\, \varphi$}, \intens{$\exists x\,\varphi$} sunt formule
\end{itemize}

\medskip \pause
\textbf{\color{True}Exemplu:}
\[P(f(x_1,c)),\quad P(x_1) \vee P(c), \quad \forall x_1\, P(x_1),  \quad \forall x_2\, R(x_2,x_1)\]
unde $c \in \mathbf{C}$, $x_1,x_2 \in V$, $f \in \mathbf{F}_2$, $P \in \mathbf{R}_1$, iar $R \in \mathbf{R}_2$.
\end{frame}
%---------------------------------------------------------------------

\frame{
\frametitle{Exercițiu}


Fie alfabetul definit prin ${\mathbf R}=\{<\}$, ${\mathbf F}=\{s, +\}$, ${\mathbf C}=\{0\}$ și\\
$ari(s)=1$, $ari(+)=ari(<)=2$.

Dați exemple de $3$ termeni, $3$ formule atomice și $3$ formule.

\pause \medskip 

Exemple de  \intens{termeni}:  

\hspace{.5cm} $0$, $x$, $s(0)$, $s(s(0))$, $s(x)$, $s(s(x))$, $\ldots$,

\hspace{.5cm} $+(0,0)$, $+(s(s(0)),+(0,s(0)))$,   
  $+(x,s(0))$, $+(x,s(x))$, $\ldots$, 

\medskip 

Exemple de \intens{formule atomice}: 

  
\hspace{.5cm}   $<(0,0)$, $<(x,0)$, 
$<(s(s(x)),s(0))$, $\ldots$

\medskip 

Exemple de \intens{formule}:

\hspace{.5cm} $\forall x\,\forall y \, <(x, +(x,y))$, $\forall x \, <(x,s(x))$

}

%---------------------------------------------------------------------
\begin{frame}{Literali }


Un \intens{literal} este o \intens{formulă atomică} sau \intens{negația unei formule atomice}.

\vfill

O formulă este în \intens{formă normală conjunctivă (FNC)} dacă este o \intens{conjuncție} de \intens{disjuncții de literali}. 

\textbf{\color{True} Exemplu:} $(P(f(x_1,c)) \vee R(c,x_2)) \ \wedge \ \neg R(x_1,x_2)\ \wedge\ (R(x_1,x_1) \vee \neg P(c))$ \\
unde $c \in \mathbf{C}$, $x_1,x_2 \in V$, $f \in \mathbf{F}_2$, $P \in \mathbf{R}_1$, iar $R\in\mathbf{R}_2$. 
\end{frame}

%---------------------------------------------------------------------
\begin{frame}{Clauze}

\begin{itemize}
\item  O \intens{clauză} este o \intens{ disjuncție de literali}.

\medskip 

\item Dacă $L_1,\ldots, L_n$ sunt literali atunci vom reprezenta clauza
$L_1\vee\ldots\vee L_n$ ca mulțimea
$\{L_1,\ldots, L_n\}$

\begin{center}
\intens{clauză = mulțime de literali} 
\end{center}


%\medskip 
%\item Clauza $C=\{L_1,\ldots, L_n\}$ este \intens{satisfiabilă}
%dacă  $L_1\vee\ldots\vee L_n$ este satisfiabilă.

\medskip 
\item O clauză $C$ este \intens{trivială} dacă conține un literal și complementul lui. 

\medskip 
\item Când $n=0$ obținem \intens{clauza vidă}, care se notează \intens{$\Box$}. 


%\medskip 
%
%\item Prin definiție, \intens{clauza $\Box$ nu este satisfiabilă.}


\end{itemize}
 
%\begin{center}
%\intens{Rezoluția este o metodă de verificare a satisfiabilității\\ unei mulțimi de clauze.}
%\end{center}
\end{frame}




\begin{frame}{Clauze}
Putem reprezenta o clauză prin mulțimea
\intens{\[\{\neg Q_1,\ldots, \neg Q_n, P_1,\ldots,  P_k\}\]}
unde $n,k\geq 0$ și $Q_1,\ldots, Q_n,P_1,\ldots, P_k$ sunt formule atomice.

\bigskip
Formula corespunzătoare este 
\[\forall x_1\ldots \forall x_m (\neg Q_1\vee\ldots \vee \neg Q_n\vee  P_1\vee \ldots\vee P_k)\]
unde $x_1,\ldots, x_m$ sunt toate variabilele care apar în clauză 

\bigskip
Echivalent, putem scrie 
\intens{\[\forall x_1\ldots \forall x_m(Q_1\wedge \ldots \wedge Q_n \to P_1\vee \ldots\vee  P_k)\]}

\vspace{-.4cm}
Presupunem cuantificarea universală a clauzelor implicită:
\intens{\[Q_1\wedge \ldots \wedge Q_n \to P_1\vee \ldots\vee  P_k \]}


\end{frame}

%---------------------------------------------------------------------

\begin{frame}{Clauze program definite}

\begin{itemize}
\item Clauză  \intens{$Q_1\wedge \ldots \wedge Q_n \to P_1\vee \ldots\vee  P_k$}\\
unde $n,k\geq 0$ și  $Q_1,\ldots, Q_n,P_1,\ldots, P_k$ sunt formule atomice.

\bigskip   

\item Dacă $k=1$, atunci avem o \intens{clauză program definită}:

\begin{itemize}
\item cazul $n>0:$ \hspace*{0.3cm}  \intens{$ Q_1\wedge \ldots\wedge  Q_n\to P$}
\smallskip 
\item cazul $n=0:$ \hspace*{0.3cm}   \intens{$\top\to P$} (clauză unitate, fapt)  \\
$\top$ este simbol pentru o formula mereu adevărată
\smallskip 
\end{itemize}

\bigskip
\item \intens{Program logic definit = mulțime finită de clauze definite}

\end{itemize}

\end{frame}

%---------------------------------------------------------------------

\begin{frame}{Clauze Horn}


\begin{itemize}

\item Clauză  \intens{$Q_1\wedge \ldots \wedge Q_n \to P_1\vee \ldots\vee  P_k$}\\
unde $n,k\geq 0$ și  $Q_1,\ldots, Q_n,P_1,\ldots, P_k$ sunt formule atomice.

\bigskip
\item Dacă $k=0$, atunci avem o \intens{clauză scop definită} (țintă, întrebare):
\begin{center}
	\intens{$Q_1\wedge\ldots\wedge Q_n\to \bot$} \\
	$\bot$ este simbol pentru o formula mereu falsă
\end{center}
Vom scrie o clauza scop definită ca \intens{$Q_1, \ldots, Q_n$}. 
\bigskip
\item În plus, dacă $n=k=0$, atunci avem \intens{clauza vidă} \intens{$\Box$}


\end{itemize}

\begin{center}
\intens{Clauză Horn = clauză program definită sau clauză scop ($k\leq 1$)}
\end{center}


\begin{center}

Limbajul \intens{PROLOG} are la bază \intens{logica clauzelor Horn}.

\end{center}
\end{frame}

%---------------------------------------------------------------------

\begin{frame}{Clauze Horn}


\begin{center}
\intens{Clauza Horn = clauză program definită sau clauză scop ($k\leq 1$)}
\end{center}


\begin{itemize}

\item Clauză  \intens{$Q_1\wedge \ldots \wedge Q_n \to P_1\vee \ldots\vee  P_k$}\\
unde $n,k\geq 0$ și  $Q_1,\ldots, Q_n,P_1,\ldots, P_k$ sunt formule atomice.


\bigskip
\item Fie $x_1, \ldots, x_m$ toate variabilele care apar într-o clauza scop 
$Q_1,\ldots, Q_n$. Atunci avem echivalența

\vspace{-.6cm}
$$
\forall x_1\ldots \forall x_m (\neg Q_1\vee\ldots \vee \neg Q_n) 
\equiv \neg \exists x_1\ldots \exists x_m (Q_1\wedge \ldots \wedge  Q_n)
$$

\end{itemize}

\begin{center}
\intens{Negația unei "întrebări" în PROLOG este clauză scop.} 


%Limbajul \intens{PROLOG} are la bază \intens{logica clauzelor Horn}.

\end{center}
\end{frame}

%---------------------------------------------------------------------
\begin{frame}{Programare logica}

\begin{itemize}
\item \intens{Logica clauzelor Horn:} un fragment al logicii de ordinul I în care singurele formule admise sunt \intens{clauze Horn}
	\begin{itemize}
		\item \intens{formule atomice}: $P(t_1,\ldots,t_n)$
		\item \intens{$Q_1\wedge \ldots \wedge Q_n \to P$} 
		
		unde toate $Q_i,P$ sunt formule atomice, $\top$ sau $\bot$
	
	\end{itemize}
	\medskip
	

	 
	\item \intens{Problema programării logice:}   reprezentăm cunoștințele ca o mulțime de clauze definite $KB$ și suntem interesați să aflăm răspunsul la o întrebare de forma $Q_1\wedge\ldots \wedge Q_n$, unde toate $Q_i$ sunt formule atomice

	\begin{center}
\intens{	$KB \models Q_1\wedge\ldots\wedge Q_n$}
	\end{center}
	
\begin{itemize}	
	\item Variabilele din $KB$  sunt  \intens{cuantificate universal}.

	\medskip
	\item Variabilele din $Q_1, \ldots,  Q_n$ sunt \intens{cuantificate existențial}.
	\end{itemize}
\end{itemize}

\end{frame}

%---------------------------------------------------------------------
\begin{frame}{Un exemplu}

Fie următoarele clauze definite:

\hspace{.4cm} $father(jon,ken)$.

\hspace{.4cm} $father(ken,liz)$.

\hspace{.4cm} $father(X,Y) \to ancestor(X,Y)$

\hspace{.4cm} $daugther(X,Y) \to ancestor(Y,X)$ 

\hspace{.4cm} $ancestor(X,Y) \wedge ancestor(Y,Z) \to ancestor(X,Z)$

\medskip
Putem pune întrebările:
\begin{itemize}
	\item \intens{$ancestor(jon,liz)$}?
	\item \intens{$ancestor(ken,Z)$}? \\(există $Z$ astfel încât $ancestor(ken,Z)$) 
\end{itemize}

\end{frame}
%-------------------------------------------------------------------

%---------------------------------------------------------------------
\begin{frame}{Sistem de deducție pentru logica clauzelor Horn}


Pentru un program logic definit $KB$  avem
\begin{itemize}
	 
	\item \intens{Axiome:} orice clauză din $KB$ 
	 
	\item \intens{Regula de deducție \textit{backchain}:}
	\[
	\infer{\theta(Q)}{\theta(Q_1)\quad \theta(Q_2) \quad \ldots \quad \theta(Q_n) \quad (Q_1 \wedge Q_2 \wedge \ldots \wedge Q_n \to P)}
	\]
	unde $Q_1 \wedge Q_2 \wedge \ldots \wedge Q_n \to P \in KB$, iar $\theta$ cmgu pentru $Q$ și $P$.
\end{itemize}

\vspace{.4cm}
Regula \textit{backchain} conduce la un \intens{sistem de deducție complet}:
\begin{center}
	
	Pentru o mulțime de clauze $KB$ și o țintă $Q$, \\
	dacă $KB \models Q$, \\
	atunci există o derivare a lui $Q$ folosind regula \textit{backchain}.
\end{center} 
\end{frame}

%---------------------------------------------------------------------
\begin{frame}{Cum răspundem la întrebări}

Pentru o țintă $Q$, trebuie să găsim o clauză din $KB$ 
\vspace{-.2cm}
\begin{center}
$Q_1 \wedge \ldots \wedge Q_n \to P$,
\end{center}
\vspace{-.2cm}

și un unificator $\theta$ pentru $Q$ și $P$. 

 În continuare vom verifica $\theta(Q_1),\ldots,\theta(Q_n)$.

\medskip  
\textbf{\color{True} Exemplu.} Pentru ținta
\[\intens{ancestor(ken,Z)},\] 
\vspace{-.2cm}
putem folosi clauză
\[\intens{father(X,Y) \to ancestor(X,Y)}\]
\vspace{-.2cm}
cu unificatorul
\vspace{-.1cm}
\[ \intens{\{ X \mapsto ken, Y \mapsto Z\} }\]
\vspace{-.2cm}
pentru a obține o nouă țintă
\[ \intens{father(ken,Z) }.\]
\vspace{-.6cm}

\end{frame}

%---------------------------------------------------------------------
\begin{frame}{Sistem de deducție}

\intens{Rergula backchain}
{\small
	\[
	\infer{\theta(Q)}{\theta(Q_1)\quad \theta(Q_2) \quad \ldots \quad \theta(Q_n) \quad (Q_1 \wedge Q_2 \wedge \ldots \wedge Q_n \to P)}
	\]
	unde $Q_1 \wedge Q_2 \wedge \ldots \wedge Q_n \to P \in KB$, iar $\theta$ este cgu pentru $Q$ și $P$.}

\medskip 


\textbf{\color{True} Exemplu.} Presupunem că în KB avem:

\hspace{.4cm} $father(ken,liz)$.

\hspace{.4cm} $father(X,Y) \to ancestor(X,Y)$

	\[
	\infer{\intens{ancestor(ken,Z)}}{\infer{\intens{father(ken,Z)}}{{father(ken,liz)}} \quad father(X,Y) \to ancestor(X,Y)}
	\]



\end{frame}


%---------------------------------------------------------------------
\begin{frame}{Puncte de decizie în programarea logica}
Având doar această regulă, care sunt punctele de decizie în căutare?
\begin{itemize}
	 
	\item \intens{Ce clauză să alegem.}
	\begin{itemize}
		\item Pot fi mai multe clauze a căror parte dreaptă se potrivește cu o țintă.
		\item Aceasta este o alegere de tip \textbf{SAU}: este suficient ca oricare din variante să reușească.
	\end{itemize}
	
	\medskip  
	\item \intens{Ordinea în care rezolvăm noile ținte.}
	\begin{itemize}
		\item Aceasta este o alegere de tip \textbf{ȘI:} trebuie arătate toate țintele noi.
		\item Ordinea în care le rezolvăm poate afecta găsirea unei derivări, depinzând de strategia de căutare folosită.
	\end{itemize}
\end{itemize}
\end{frame}
%---------------------------------------------------------------------

%---------------------------------------------------------------------
\begin{frame}{Strategia de căutare din Prolog}

Strategia de căutare din Prolog este de tip \intens{\textit{depth-first}}
\medskip
\begin{itemize}
	\item \intens{de sus în jos}
	\begin{itemize}
		\item pentru alegerile de tip \textbf{SAU}
		\item alege clauzele în ordinea în care apar în program
	\end{itemize}
	\medskip
	\item \intens{de la stânga la dreapta}
	\begin{itemize}
		\item pentru alegerile de tip \textbf{ȘI}
		\item alege noile ținte în ordinea în care apar în clauza aleasă
	\end{itemize}
\end{itemize}

\end{frame}
%---------------------------------------------------------------------


%
%%---------------------------------------------------------------------
%\begin{frame}{Sistemul de inferență backchain}
%
%Notăm cu \intens{$KB \vdash_b Q$} dacă există o derivare a lui $Q$ din $KB$ folosind sistemul de inferență {\em backchain}.
%
%\medskip
%\begin{theorem}
%Sistemul de inferență {\em backchain} este \intens{corect} și \intens{complet} pentru formule atomice fără variabile $Q$.
%\begin{center}
%\intens{$KB \models Q$} \quad \intens{dacă și numai dacă} \quad \intens{$KB \vdash_b Q$}
%\end{center}
%\end{theorem}
%
%\medskip  
%Sistemul de inferență {\em backchain} este \intens{corect} și \intens{complet} și pentru formule atomice \intens{cu variabile} $Q$: 
%\begin{center}
%\intens{$KB \models \exists x Q(x)$}  {dacă și numai dacă} \intens{$KB \vdash_b \theta(Q)$ } \\
%\hspace*{6.5cm}\intens{pentru o substituție $\theta$.}
%\end{center}
%
%\end{frame}
%%---------------------------------------------------------------------
%


%---------------------------------------------------------------------
\begin{frame}{Regula {backchain} și rezoluția SLD}
\begin{itemize}
	\item Regula \textit{backchain} este implementată în programarea logică prin \intens{rezoluția SLD} (Selected, Linear, Definite).
	
	\medskip
	\item Prolog are la bază rezoluția SLD.
\end{itemize}
\end{frame}
%---------------------------------------------------------------------


%------------------------------------------------------------------------
\begin{frame}{Rezoluția SLD}
 
\vspace{.2cm}
Fie $KB$ o mulțime de clauze definite.

\medskip
\begin{center}
\intens{SLD}
\fbox{
$\displaystyle\frac{\neg Q_1\vee\cdots\vee \intens{\neg Q_i}\vee\cdots\vee\neg Q_n }
{\intens{ \theta}(\neg Q_1\vee\cdots\vee \intens{\neg P_1\vee\cdots\vee\neg P_m}\vee\cdots\vee\neg Q_n)}$}

\end{center}
\vspace{-.2cm}
unde 
\begin{itemize}
	\item \intens{ $Q\vee\neg P_1\vee\cdots\vee \neg P_m$} este o clauză definită din $KB$
	
	\item în care toate variabilele au fost redenumite cu variabile noi

	\item \intens{ $\theta$} este cmgu pentru \intens{ $Q_i$} și \intens{ $Q$}

\end{itemize}
\end{frame}

%------------------------------------------------------------------
\begin{frame}{Rezoluția SLD - exemplu}

{
\footnotesize
\vspace{.2cm}
\begin{minipage}{7cm}
\begin{alltt}
father(eddard,sansa). \\
father(eddard,jonSnow). \\

stark(eddard). \\
stark(catelyn). \\

stark(X) :- father(Y,X),    
    stark(Y).
\end{alltt}
\end{minipage}
\begin{minipage}{3.5cm}
\texttt{?- stark(jonSnow)}
\end{minipage}

\vspace{.2cm}
\begin{center}
\intens{SLD}
\fbox{
$\displaystyle\frac{\neg Q_1\vee\cdots\vee \intens{\neg Q_i}\vee\cdots\vee\neg Q_n }
{\intens{ \theta}(\neg Q_1\vee\cdots\vee \intens{\neg P_1\vee\cdots\vee\neg P_m}\vee\cdots\vee\neg Q_n)}$}
\end{center}
\vspace{-.3cm}
\begin{itemize}
	\item \intens{ $Q\vee\neg P_1\vee\cdots\vee \neg P_m$} este o clauză definită din $KB$
	\item variabilele din \intens{$Q\vee\neg P_1\vee\cdots\vee \neg P_m$} și \intens{$Q_i$} se redenumesc
	\item \intens{ $\theta$} este cmgu pentru \intens{ $Q_i$} și \intens{ $Q$}.
\end{itemize}
}
\end{frame}



%------------------------------------------------------------------
\begin{frame}{Rezoluția SLD - exemplu}

{
\footnotesize
\vspace{.2cm}
\begin{minipage}{6.5cm}
$father(eddard,sansa)$ \\
$father(eddard,jonSnow)$ \\

$stark(eddard)$ \\
$stark(catelyn)$ \\


\intens{$stark(X)\vee\neg father(Y,X)\vee\neg stark(Y)$}
\end{minipage}
\begin{minipage}{3.5cm}
$$\displaystyle\frac{\onslide<1->{\intens{\neg stark(jonSnow)}}}
{\onslide<2->{\intens{\neg father(Y,jonSnow)\vee \neg stark(Y)}}}$$

\onslide<1->{$$\theta(X) = jonSnow$$}
\end{minipage}


\vspace{.2cm}
\begin{center}
\intens{SLD}
\fbox{
$\displaystyle\frac{\neg Q_1\vee\cdots\vee \intens{\neg Q_i}\vee\cdots\vee\neg Q_n }
{\intens{ \theta}(\neg Q_1\vee\cdots\vee \intens{\neg P_1\vee\cdots\vee\neg P_m}\vee\cdots\vee\neg Q_n)}$}
\end{center}
\vspace{-.3cm}
\begin{itemize}
	\item \intens{ $Q\vee\neg P_1\vee\cdots\vee \neg P_m$} este o clauză definită din $KB$
	\item variabilele din \intens{$Q\vee\neg P_1\vee\cdots\vee \neg P_m$} și \intens{$Q_i$} se redenumesc
	\item \intens{ $\theta$} este cmgu pentru \intens{ $Q_i$} și \intens{ $Q$}.
\end{itemize}
}
\end{frame}

%------------------------------------------------------------------
\begin{frame}{Rezoluția SLD - exemplu}

{
\footnotesize
\vspace{.2cm}
\begin{minipage}{6.5cm}
$father(eddard,sansa)$ \\
$father(eddard,jonSnow)$ \\

$stark(eddard)$ \\
$stark(catelyn)$ \\


\intens{$stark(X)\vee\neg father(Y,X)\vee\neg stark(Y)$}
\end{minipage}
\begin{minipage}{3.5cm}
$$\displaystyle\frac{{\intens{\neg stark(jonSnow)}}}
{{\intens{\neg father(Y,jonSnow)\vee \neg stark(Y)}}}$$
 
$$\displaystyle\frac
{{\intens{\neg father(Y,jonSnow)}\vee \neg stark(Y)}} {{ \neg stark(eddard)}}$$
 

$$\displaystyle\frac {\intens{ \neg stark(eddard)}}{\Box}$$
\end{minipage}


\vspace{.2cm}
\begin{center}
\intens{SLD}
\fbox{
$\displaystyle\frac{\neg Q_1\vee\cdots\vee \intens{\neg Q_i}\vee\cdots\vee\neg Q_n }
{\intens{ \theta}(\neg Q_1\vee\cdots\vee \intens{\neg P_1\vee\cdots\vee\neg P_m}\vee\cdots\vee\neg Q_n)}$}
\end{center}
\vspace{-.3cm}
\begin{itemize}
	\item \intens{ $Q\vee\neg P_1\vee\cdots\vee \neg P_m$} este o clauză definită din $KB$
	\item variabilele din \intens{$Q\vee\neg P_1\vee\cdots\vee \neg P_m$} și \intens{$Q_i$} se redenumesc
	\item \intens{ $\theta$} este cmgu pentru \intens{ $Q_i$} și \intens{ $Q$}.
\end{itemize}
}
\end{frame}

%------------------------------------------------------------------------
\begin{frame}{Rezoluția SLD }
 
Fie $KB$ o mulțime de clauze definite și $Q_1 \wedge \ldots \wedge Q_m$ o întrebare, unde $Q_i$ sunt formule atomice.

\begin{itemize}
	\item \intens{O derivare} din $KB$ prin rezoluție SLD este o secvență 
\begin{center}
\intens{$G_0:= \neg Q_1 \vee \ldots \vee \neg Q_m$},\quad  $G_1$,\quad $\ldots$,\quad $G_k$, $\ldots$
\end{center}
în care $G_{i+1}$ se obține din $G_i$ prin regula \intens{ SLD}.  

	\item Dacă există un $k$ cu 
\intens{$G_k=\Box$} (clauza vidă), atunci  derivarea se numește \intens{SLD-respingere}.
\end{itemize}

\pause
\textbf{Teoremă.} Sunt echivalente:
\begin{enumerate}
\item există o \intens{SLD-respingere} a lui $Q_1 \wedge \ldots \wedge Q_m$ din  $KB$,
\vspace{.2cm}
\item  \intens{$KB\models Q_1\wedge\cdots\wedge Q_m$}.
\end{enumerate}
\end{frame}


%------------------------------------------------------------------------
\begin{frame}{Rezoluția SLD - arbori de căutare}
 \intens{Arbori SLD}
 \begin{itemize}
 	\item Presupunem că avem o mulțime de clauze definite $KB$ și o țintă $G_0 = \neg Q_1 \vee \ldots \vee \neg Q_m$
	\medskip
	\item Construim un arbore de căutare (\intens{arbore SLD}) astfel:
	\begin{itemize}
		\vspace{.1cm}
		\item Fiecare nod al arborelui este o țintă (posibil vidă)
		\vspace{.1cm}
		\item Rădăcina este $G_0$
		\vspace{.1cm}
		\item Dacă arborele are un nod $G_i$, iar $G_{i+1}$ se obține din $G_i$ folosind regula SLD folosind o clauză $C_i \in KB$, atunci nodul  $G_i$ are copilul $G_{i+1}$. \\
		Muchia dintre $G_i$ și $G_{i+1}$ este etichetată cu $C_i$.
	\end{itemize}
	\medskip
	\item Dacă un arbore SLD cu rădăcina $G_0$ are o frunză $\Box$ (clauza vidă), atunci există o SLD-respingere a lui $G_0$ din $KB$.
\end{itemize}
\end{frame}

%------------------------------------------------------------------------
\begin{frame}{Rezoluția SLD - arbori de căutare}

\textbf{\color{True} Exemplu.}

Fie $KB$ următoarea mulțime de clauze definite:
\begin{enumerate}
	\item \intens{$grandfather(X,Z):-father(X,Y), parent(Y,Z)$}
	\item \intens{$parent(X,Y):-father(X,Y)$}
	\item \intens{$parent(X,Y):-mother(X,Y)$}
	\item \intens{$father(ken,diana)$}
	\item \intens{$mother(diana,brian)$}
\end{enumerate}

\vspace{.2cm}
\intens{Găsiți o respingere din $KB$ pentru}
\begin{center}
\intens{$?- grandfather(ken,Y)$}
\end{center}
\end{frame}

%------------------------------------------------------------------------
\begin{frame}{Rezoluția SLD - arbori de căutare}

\textbf{\color{True} Exemplu.}

Fie $KB$ următoarea mulțime de clauze definite:
\begin{enumerate}
	\item \intens{$grandfather(X,Z) \vee \neg father(X,Y) \vee \neg parent(Y,Z)$}
	\item \intens{$parent(X,Y) \vee \neg father(X,Y)$}
	\item \intens{$parent(X,Y) \vee \neg mother(X,Y)$}
	\item \intens{$father(ken,diana)$}
	\item \intens{$mother(diana,brian)$}
\end{enumerate}

\vspace{.2cm}
\intens{Găsiți o respingere din $KB$ pentru}
\begin{center}
\intens{$\neg grandfather(ken,Y)$}
\end{center}

\end{frame}

%------------------------------------------------------------------------
\begin{frame}{Rezoluția SLD - arbori de căutare}

{\footnotesize

\textbf{\color{True} Exemplu.}

\begin{enumerate}
	\item {$grandfather(X,Z) \vee \neg father(X,Y)\vee \neg parent(Y,Z)$}
	\item {$parent(X,Y)\vee \neg father(X,Y)$}
	\item {$parent(X,Y)\vee \neg mother(X,Y)$}
	\item {$father(ken,diana)$}
	\item {$mother(diana,brian)$}
\end{enumerate}

\vspace{-.1cm}
\begin{figure}[h]
  \begin{tikzpicture}	
  	\node(1) at (0,4) {\intens{$\neg grandfather(ken,Y)$}};
	\node(2) at (0,3) {\intens{$\neg father(ken,V)$} $\vee\ \neg parent(V,Y)$};
	\node(3) at (0,2) {\intens{$\neg parent(diana,Y)$}};
	\node(4) at (-2,1) {$\neg father(diana,Y)$};
	\node(5) at (2,1) {\intens{$\neg mother(diana,Y)$}};
	\node(6) at (2,0) {$\Box$};
	  	
	\draw [->]  (1) -- node[midway,right] {\footnotesize $1$}  (2) ;
	\draw [->]  (2) -- node[midway,right] {\footnotesize $4$}  (3) ;
	\draw [->]  (3) -- node[midway,right] {\footnotesize $2$}  (4) ;
	\draw [->]  (3) -- node[midway,right] {\footnotesize $3$}  (5) ;
	\draw [->]  (5) -- node[midway,right] {\footnotesize $5$}  (6) ;
   \end{tikzpicture}
\end{figure}
}
\end{frame}

%%------------------------------------------------------------------------
%\begin{frame}{Rezoluția SLD }
%
%\vspace*{.3cm}
%
%\begin{example} 
%\begin{minipage}{5.5cm}
%{
%\begin{enumerate}
% \setcounter{enumi}{1}
%
%	\item {$\intens{parent(X,Y)}\vee \neg father(X,Y)$}
%	\end{enumerate}
%}
%\end{minipage}
%\begin{minipage}{4.5cm}
%\begin{figure}[h]
%  \begin{tikzpicture}	
%	\node(3) at (0,2) {\intens{$\neg parent(diana,Y)$}};
%	\node(4) at (-2,1) {$\neg father(diana,Y)$};
%	\draw [->]  (3) -- node[midway,right] {\footnotesize $2$}  (4) ;
%
%   \end{tikzpicture}
%  \end{figure}
%  \end{minipage}
%
%\medskip
%Aplicarea \intens{SLD}:
%\medskip 
%\begin{itemize}
%\item redenumesc variabilele: $\intens{parent(X,Y_2)}\vee \neg father(X,Y_2)$
%\medskip
%\item determin unificatorul: \intens{$\theta={X/diana, Y_2/Y}$}
%\medskip
%\item aplic regula: $\displaystyle\frac {\intens{\neg parent(diana,Y)}}{\intens{\neg father(diana,Y)}}$
%\end{itemize}
%\end{example}
%\end{frame}

%---------------------------------------------------------------------
\begin{frame}{Limbajul Prolog}
\begin{itemize}
	\item Am arătat că \intens{sistemul de inferență din spatele Prolog-ului este complet}. 
	\begin{itemize}
		\item Dacă o întrebare este consecință logică a unei mulțimi de clauze, atunci există o derivare a întrebării.
	\end{itemize}
	\medskip
	\item Totuși, \alert{strategia de căutate din Prolog este incompletă!}
	\begin{itemize}
		\item Chiar dacă o întrebare este consecință logică a unei mulțimi de clauze, Prolog nu găsește mereu o derivare a întrebării.
	\end{itemize}
	
\end{itemize}
\end{frame}
%---------------------------------------------------------------------


%---------------------------------------------------------------------
\begin{frame}{Limbajul Prolog - exemplu}

\begin{alltt}
warmerClimate :- albedoDecrease. \\
warmerClimate :- carbonIncrease. \\
iceMelts :- warmerClimate. \\
albedoDecrease :- iceMelts. \\
carbonIncrease. \\

?- iceMelts.\\
\alert{! Out of local stack}
\end{alltt}

\end{frame}


\begin{frame}{Limbajul Prolog - exemplu}

\begin{alltt}
\alert{warmerClimate} :- \alert{albedoDecrease.} \\
warmerClimate :- carbonIncrease. \\
\alert{iceMelts} :- \alert{warmerClimate}. \\
\alert{albedoDecrease} :- \alert{iceMelts.} \\
carbonIncrease. \\

?- iceMelts.\\
\alert{! Out of local stack}
\end{alltt}

\end{frame}

%---------------------------------------------------------------------
\begin{frame}{Limbajul Prolog - exemplu}

Totuși, există o derivare a lui $iceMelts$ în sistemul de deducție din clauzele:
\begin{center}
\begin{tabular}{rcl}

$albedoDecrease$ & $\to$ & $warmerClimate  $ \\
$carbonIncrease$ & $\to$ & $warmerClimate $ \\
$warmerClimate$ & $\to$ & $iceMelts$  \\
$iceMelts$ & $\to$ & $albedoDecrease$ \\
$\top$&$\to$ & $carbonIncrease $ \\

\end{tabular}

\bigskip
$
\inferrule
	{\inferrule
	{carbonInc. \\ carbonInc. \to warmerClim.}
	{warmerClim.} \\ warmerClim. \to iceMelts}
	{iceMelts}
$
\end{center}

\end{frame}

%---------------------------------------------------------------------

\begin{frame}{Rezoluția SLD - arbori de căutare}

\textbf{\color{True} Exercițiu}
Desenați arborele SLD pentru programul Prolog de mai jos și ținta \\\texttt{?- p(X,X).}

\medskip

\begin{tabular}{ll}
	\texttt{1. p(X,Y) :- q(X,Z), r(Z,Y).} & \hspace{.5cm} \texttt{7. s(X) :- t(X,a).}  \\
	\texttt{2. p(X,X) :- s(X).}		  & \hspace{.5cm} \texttt{8. s(X) :- t(X,b).}  \\
	\texttt{3. q(X,b).}  			  & \hspace{.5cm} \texttt{9. s(X) :- t(X,X).}  \\
	\texttt{4. q(b,a).} 			  & \hspace{.5cm} \texttt{10. t(a,b).}  \\
	\texttt{5. q(X,a) :- r(a,X).} 		  & \hspace{.5cm} \texttt{11. t(b,a).}  \\
	\texttt{6. r(b,a).} 			  &
\end{tabular}

\end{frame}

\begin{frame}{Rezoluția SLD - arbori de căutare}

\tiny{\begin{tabular}{llll}
	\texttt{1. p(X,Y) :- q(X,Z), r(Z,Y).} & \hspace{.3cm} \texttt{4. q(b,a).}  &\hspace{.3cm} \texttt{7. s(X) :- t(X,a).}  & \hspace{.3cm} \texttt{10. t(a,b).}\\
\texttt{2. p(X,X) :- s(X).}		  &  \hspace{.3cm} \texttt{5. q(X,a) :- r(a,X).}
	&\hspace{.3cm} \texttt{8. s(X) :- t(X,b).}  &\hspace{.3cm}\texttt{11. t(b,a).} \\
	\texttt{3. q(X,b).}  			  & 
	\hspace*{.3cm} \texttt{6. r(b,a).} &\hspace{.3cm} \texttt{9. s(X) :- t(X,X).} &  \\
	 & & & \\
			$p(X,Y) \vee \neg q(X,Z)\vee\neg r(Z,Y)$ & \hspace{.3cm} $ q(b,a)$  &\hspace{.3cm} $s(X) \vee\neg t(X,a)$  & \hspace{.3cm}  $t(a,b)$\\
		$p(X,X) \vee\neg s(X)$		  &  \hspace{.3cm} $q(X,a) \vee\neg r(a,X)$
	&\hspace{.3cm} $s(X) \vee\neg t(X,b)$  &\hspace{.3cm}$t(b,a)$ \\
	
		$ q(X,b)$  			  & 
	\hspace*{.3cm} $ r(b,a)$ &\hspace{.3cm} 
	$s(X) \vee\neg t(X,X)$ &  
\end{tabular}}

\bigskip

\begin{tikzpicture}[
  every node/.style = {align=center,font=\small}]
\node (n1) {$\neg p(X,X)$};
\node [below left=15pt of n1] (n11){$\neg q(X,Z)\vee\neg r(Z,X)$};
\node [below =15pt of n11] (n113){$\neg r(a,X)$};
\node [below =0.5pt of n113] (nf2){(fail)};
\node [left=2pt and 1pt of n113] (n112){$\neg r(a,b)$};
\node [below=0.5pt of n112] (nf1){$(fail)$};
\node [left=2pt of n112] (n111){$\neg r(b,X)$};
\node [below=15pt of n111] (n1111){$\Box$\\(X=a)};
\node [right=2pt of n113] (n114){$\neg q(a,b)$};
\node [below=15pt of n114] (n1141){$\Box$\\(X=a)};
\node [below right=15pt and 15pt of n1] (n12){$\neg s(X)$};
\node [below=15pt of n12] (n122){$\neg t(X,b)$};
\node [left=2pt of n122] (n121){$\neg t(X,a)$};
\node [right=2pt of n122] (n123){$\neg t(X,X)$};
\node [below=0.5pt of n123] (nf11){$(fail)$};
\node [below=15pt of n121] (n1211){$\Box$\\(X=b)};
\node [below=15pt of n122] (n1221){$\Box$\\(X=a)};

\draw [-]  (n1) -- node[midway,left] {\footnotesize $1$}  (n11) ;
\draw [-]  (n1) -- node[midway,right] {\footnotesize $2$}  (n12) ;
\draw [-]  (n11) -- node[midway,left] {\footnotesize $3$}  (n111) ;
\draw [-]  (n11) -- node[midway,left] {\footnotesize $4$}  (n112) ;
\draw [-]  (n11) -- node[midway,left] {\footnotesize $5$}  (n113) ;
\draw [-]  (n11) -- node[midway,right] {\footnotesize $6$}  (n114) ;
\draw [-]  (n12) -- node[midway,left] {\footnotesize $7$}  (n121) ;
\draw [-]  (n12) -- node[midway,left] {\footnotesize $8$}  (n122) ;
\draw [-]  (n12) -- node[midway,right] {\footnotesize $9$}  (n123) ;
\draw [-]  (n121) -- node[midway,right] {\footnotesize $11$}  (n1211) ;
\draw [-]  (n122) -- node[midway,right] {\footnotesize $10$}  (n1221) ;
\draw [-]  (n111) -- node[midway,right] {\footnotesize $6$}  (n1111) ;
\draw [-]  (n114) -- node[midway,right] {\footnotesize $3$}  (n1141) ;

\end{tikzpicture}

\end{frame}


%---------------------------------------------
\begin{frame}
  \vfill
  \centering

\textbf{Pe data viitoare!}

  \vfill
\end{frame}

\end{document}







